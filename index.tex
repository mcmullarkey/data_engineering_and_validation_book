% Options for packages loaded elsewhere
\PassOptionsToPackage{unicode}{hyperref}
\PassOptionsToPackage{hyphens}{url}
\PassOptionsToPackage{dvipsnames,svgnames,x11names}{xcolor}
%
\documentclass[
  letterpaper,
  DIV=11,
  numbers=noendperiod]{scrreprt}

\usepackage{amsmath,amssymb}
\usepackage{lmodern}
\usepackage{iftex}
\ifPDFTeX
  \usepackage[T1]{fontenc}
  \usepackage[utf8]{inputenc}
  \usepackage{textcomp} % provide euro and other symbols
\else % if luatex or xetex
  \usepackage{unicode-math}
  \defaultfontfeatures{Scale=MatchLowercase}
  \defaultfontfeatures[\rmfamily]{Ligatures=TeX,Scale=1}
\fi
% Use upquote if available, for straight quotes in verbatim environments
\IfFileExists{upquote.sty}{\usepackage{upquote}}{}
\IfFileExists{microtype.sty}{% use microtype if available
  \usepackage[]{microtype}
  \UseMicrotypeSet[protrusion]{basicmath} % disable protrusion for tt fonts
}{}
\makeatletter
\@ifundefined{KOMAClassName}{% if non-KOMA class
  \IfFileExists{parskip.sty}{%
    \usepackage{parskip}
  }{% else
    \setlength{\parindent}{0pt}
    \setlength{\parskip}{6pt plus 2pt minus 1pt}}
}{% if KOMA class
  \KOMAoptions{parskip=half}}
\makeatother
\usepackage{xcolor}
\setlength{\emergencystretch}{3em} % prevent overfull lines
\setcounter{secnumdepth}{5}
% Make \paragraph and \subparagraph free-standing
\ifx\paragraph\undefined\else
  \let\oldparagraph\paragraph
  \renewcommand{\paragraph}[1]{\oldparagraph{#1}\mbox{}}
\fi
\ifx\subparagraph\undefined\else
  \let\oldsubparagraph\subparagraph
  \renewcommand{\subparagraph}[1]{\oldsubparagraph{#1}\mbox{}}
\fi

\usepackage{color}
\usepackage{fancyvrb}
\newcommand{\VerbBar}{|}
\newcommand{\VERB}{\Verb[commandchars=\\\{\}]}
\DefineVerbatimEnvironment{Highlighting}{Verbatim}{commandchars=\\\{\}}
% Add ',fontsize=\small' for more characters per line
\usepackage{framed}
\definecolor{shadecolor}{RGB}{241,243,245}
\newenvironment{Shaded}{\begin{snugshade}}{\end{snugshade}}
\newcommand{\AlertTok}[1]{\textcolor[rgb]{0.68,0.00,0.00}{#1}}
\newcommand{\AnnotationTok}[1]{\textcolor[rgb]{0.37,0.37,0.37}{#1}}
\newcommand{\AttributeTok}[1]{\textcolor[rgb]{0.40,0.45,0.13}{#1}}
\newcommand{\BaseNTok}[1]{\textcolor[rgb]{0.68,0.00,0.00}{#1}}
\newcommand{\BuiltInTok}[1]{\textcolor[rgb]{0.00,0.23,0.31}{#1}}
\newcommand{\CharTok}[1]{\textcolor[rgb]{0.13,0.47,0.30}{#1}}
\newcommand{\CommentTok}[1]{\textcolor[rgb]{0.37,0.37,0.37}{#1}}
\newcommand{\CommentVarTok}[1]{\textcolor[rgb]{0.37,0.37,0.37}{\textit{#1}}}
\newcommand{\ConstantTok}[1]{\textcolor[rgb]{0.56,0.35,0.01}{#1}}
\newcommand{\ControlFlowTok}[1]{\textcolor[rgb]{0.00,0.23,0.31}{#1}}
\newcommand{\DataTypeTok}[1]{\textcolor[rgb]{0.68,0.00,0.00}{#1}}
\newcommand{\DecValTok}[1]{\textcolor[rgb]{0.68,0.00,0.00}{#1}}
\newcommand{\DocumentationTok}[1]{\textcolor[rgb]{0.37,0.37,0.37}{\textit{#1}}}
\newcommand{\ErrorTok}[1]{\textcolor[rgb]{0.68,0.00,0.00}{#1}}
\newcommand{\ExtensionTok}[1]{\textcolor[rgb]{0.00,0.23,0.31}{#1}}
\newcommand{\FloatTok}[1]{\textcolor[rgb]{0.68,0.00,0.00}{#1}}
\newcommand{\FunctionTok}[1]{\textcolor[rgb]{0.28,0.35,0.67}{#1}}
\newcommand{\ImportTok}[1]{\textcolor[rgb]{0.00,0.46,0.62}{#1}}
\newcommand{\InformationTok}[1]{\textcolor[rgb]{0.37,0.37,0.37}{#1}}
\newcommand{\KeywordTok}[1]{\textcolor[rgb]{0.00,0.23,0.31}{#1}}
\newcommand{\NormalTok}[1]{\textcolor[rgb]{0.00,0.23,0.31}{#1}}
\newcommand{\OperatorTok}[1]{\textcolor[rgb]{0.37,0.37,0.37}{#1}}
\newcommand{\OtherTok}[1]{\textcolor[rgb]{0.00,0.23,0.31}{#1}}
\newcommand{\PreprocessorTok}[1]{\textcolor[rgb]{0.68,0.00,0.00}{#1}}
\newcommand{\RegionMarkerTok}[1]{\textcolor[rgb]{0.00,0.23,0.31}{#1}}
\newcommand{\SpecialCharTok}[1]{\textcolor[rgb]{0.37,0.37,0.37}{#1}}
\newcommand{\SpecialStringTok}[1]{\textcolor[rgb]{0.13,0.47,0.30}{#1}}
\newcommand{\StringTok}[1]{\textcolor[rgb]{0.13,0.47,0.30}{#1}}
\newcommand{\VariableTok}[1]{\textcolor[rgb]{0.07,0.07,0.07}{#1}}
\newcommand{\VerbatimStringTok}[1]{\textcolor[rgb]{0.13,0.47,0.30}{#1}}
\newcommand{\WarningTok}[1]{\textcolor[rgb]{0.37,0.37,0.37}{\textit{#1}}}

\providecommand{\tightlist}{%
  \setlength{\itemsep}{0pt}\setlength{\parskip}{0pt}}\usepackage{longtable,booktabs,array}
\usepackage{calc} % for calculating minipage widths
% Correct order of tables after \paragraph or \subparagraph
\usepackage{etoolbox}
\makeatletter
\patchcmd\longtable{\par}{\if@noskipsec\mbox{}\fi\par}{}{}
\makeatother
% Allow footnotes in longtable head/foot
\IfFileExists{footnotehyper.sty}{\usepackage{footnotehyper}}{\usepackage{footnote}}
\makesavenoteenv{longtable}
\usepackage{graphicx}
\makeatletter
\def\maxwidth{\ifdim\Gin@nat@width>\linewidth\linewidth\else\Gin@nat@width\fi}
\def\maxheight{\ifdim\Gin@nat@height>\textheight\textheight\else\Gin@nat@height\fi}
\makeatother
% Scale images if necessary, so that they will not overflow the page
% margins by default, and it is still possible to overwrite the defaults
% using explicit options in \includegraphics[width, height, ...]{}
\setkeys{Gin}{width=\maxwidth,height=\maxheight,keepaspectratio}
% Set default figure placement to htbp
\makeatletter
\def\fps@figure{htbp}
\makeatother

\usepackage{booktabs}
\usepackage{caption}
\usepackage{longtable}
\KOMAoption{captions}{tableheading}
\makeatletter
\makeatother
\makeatletter
\@ifpackageloaded{bookmark}{}{\usepackage{bookmark}}
\makeatother
\makeatletter
\@ifpackageloaded{caption}{}{\usepackage{caption}}
\AtBeginDocument{%
\ifdefined\contentsname
  \renewcommand*\contentsname{Table of contents}
\else
  \newcommand\contentsname{Table of contents}
\fi
\ifdefined\listfigurename
  \renewcommand*\listfigurename{List of Figures}
\else
  \newcommand\listfigurename{List of Figures}
\fi
\ifdefined\listtablename
  \renewcommand*\listtablename{List of Tables}
\else
  \newcommand\listtablename{List of Tables}
\fi
\ifdefined\figurename
  \renewcommand*\figurename{Figure}
\else
  \newcommand\figurename{Figure}
\fi
\ifdefined\tablename
  \renewcommand*\tablename{Table}
\else
  \newcommand\tablename{Table}
\fi
}
\@ifpackageloaded{float}{}{\usepackage{float}}
\floatstyle{ruled}
\@ifundefined{c@chapter}{\newfloat{codelisting}{h}{lop}}{\newfloat{codelisting}{h}{lop}[chapter]}
\floatname{codelisting}{Listing}
\newcommand*\listoflistings{\listof{codelisting}{List of Listings}}
\makeatother
\makeatletter
\@ifpackageloaded{caption}{}{\usepackage{caption}}
\@ifpackageloaded{subcaption}{}{\usepackage{subcaption}}
\makeatother
\makeatletter
\@ifpackageloaded{tcolorbox}{}{\usepackage[many]{tcolorbox}}
\makeatother
\makeatletter
\@ifundefined{shadecolor}{\definecolor{shadecolor}{rgb}{.97, .97, .97}}
\makeatother
\makeatletter
\makeatother
\ifLuaTeX
  \usepackage{selnolig}  % disable illegal ligatures
\fi
\IfFileExists{bookmark.sty}{\usepackage{bookmark}}{\usepackage{hyperref}}
\IfFileExists{xurl.sty}{\usepackage{xurl}}{} % add URL line breaks if available
\urlstyle{same} % disable monospaced font for URLs
\hypersetup{
  pdftitle={Data Engineering and Validation},
  pdfauthor={Michael Mullarkey},
  colorlinks=true,
  linkcolor={blue},
  filecolor={Maroon},
  citecolor={Blue},
  urlcolor={Blue},
  pdfcreator={LaTeX via pandoc}}

\title{Data Engineering and Validation}
\usepackage{etoolbox}
\makeatletter
\providecommand{\subtitle}[1]{% add subtitle to \maketitle
  \apptocmd{\@title}{\par {\large #1 \par}}{}{}
}
\makeatother
\subtitle{From the Ground Up}
\author{Michael Mullarkey}
\date{2023-09-27T00:00:00-06:00}

\begin{document}
\maketitle
\ifdefined\Shaded\renewenvironment{Shaded}{\begin{tcolorbox}[interior hidden, boxrule=0pt, sharp corners, enhanced, frame hidden, breakable, borderline west={3pt}{0pt}{shadecolor}]}{\end{tcolorbox}}\fi

\renewcommand*\contentsname{Table of contents}
{
\hypersetup{linkcolor=}
\setcounter{tocdepth}{2}
\tableofcontents
}
\bookmarksetup{startatroot}

\hypertarget{preface}{%
\chapter*{Preface}\label{preface}}
\addcontentsline{toc}{chapter}{Preface}

\hypertarget{why-does-this-book-exist}{%
\section*{Why Does This Book Exist?}\label{why-does-this-book-exist}}
\addcontentsline{toc}{section}{Why Does This Book Exist?}

\href{https://blog.pragmaticengineer.com/what-is-data-engineering/}{Data
engineering and validation} is the practice of systematically putting
together accurate, well-structured data.

Data engineering and validation is a
\href{https://finance.yahoo.com/news/big-data-engineering-services-market-125600472.html}{multi-billion
dollar industry}.

Data engineering and validation is absent at many smaller
organizations.\footnote{Or at least a systematic, repeatable process is
  absent}

At first glance, these statements might seem contradictory. If the
processes that help ensure accurate, usable data exude value then
wouldn't \emph{all} organizations would jump to implement them?

However, the value of data engineering and validation is precisely what
can put it out of reach for smaller organizations. Valuable tools are
expensive ones, and common data engineering and validation tools are
often outside smaller organizations' budgets.\footnote{Sure, there are
  plenty of open source tools that enable data engineering +
  validation\ldots{} and the most user-friendly ways to implement those
  tools are often in pay-to-use environments.}

Know-how of multiple indsutry-standard programming languages\footnote{Like
  Python and SQL} and technologies\footnote{Like Docker} are possible to
find within smaller organizations. And those skills are valuable enough
that retaining people with those skills is far from a guarantee.

Beyond potentially prohibitive costs for tools and talent, most data
engineering and validation systems necessary for larger-scale industry
applications are overkill for smaller organizations. Many publicly
available resources are great if your organization is processing
millions of rows of data and over-engineering if your organization is
processing thousands of rows.\footnote{Or even less!}

This resource gap is part of what drives the lack of systematic,
repeatable processes for creating accurate datasets at smaller
organizations. This book aims to help fill that gap.

\hypertarget{what-will-i-get-out-of-this-book}{%
\section*{What Will I Get Out of This
Book?}\label{what-will-i-get-out-of-this-book}}
\addcontentsline{toc}{section}{What Will I Get Out of This Book?}

We'll work on data engineering and validation from the ground up using
R, an open-source language that's familiar to many people in smaller
organizations. You won't have to touch any other programming languages
or technologies to implement everything in this book.

We'll start with how to create a pipeline using one remote
dataset,\footnote{Instead of having to download a file locally} learn
how to debug our new pipeline, add automated tests for data quality,
extend the pipeline to multiple datasets, and perform certain automated
tests on some of those datasets but not others.

This book will give you thte building blocks to create a systematic,
repeatable, and scalable,\footnote{At least to \textasciitilde hundreds
  of thousands of rows} pipeline for engineering accurate data within
the R ecosystem.

\hypertarget{who-is-this-book-for}{%
\section*{Who is This Book For?}\label{who-is-this-book-for}}
\addcontentsline{toc}{section}{Who is This Book For?}

This book is primarily geared toward people working at smaller
organizations\footnote{Think acaedmic labs, non-profits, local
  governments, etc.} with at least intermediate experience in R.

People interested in data engineering in general with a programming
background outside R might also appreciate the ``from the ground up''
approach. Other data engineering resources sometimes jump straight to a
very high complexity level or assume a computer science
background.\footnote{This book does not assume a computer science
  background} If you're looking for an introduction to data engineering
and validation at industry scale I'd recommend checking out the
\href{https://www.youtube.com/@SeattleDataGuy/videos}{Seattle Data Guy's
Youtube Videos}.

I don't want to gatekeep anyone, and this book is unlikely to be a good
way to acquaint yourself with R. If you're looking for an introductory
experience for the R programming language, I'd recommend
\href{https://psyteachr.github.io/ads-v2/}{Applied Data Skills} by Emily
Nordmann and Lisa DeBruine.

This website is and will always be free, licensed under the
\href{https://creativecommons.org/licenses/by-nc-nd/3.0/us/}{CC BY-NC-ND
3.0 License}.

\hypertarget{who-wrote-this-book}{%
\section*{Who Wrote This Book?}\label{who-wrote-this-book}}
\addcontentsline{toc}{section}{Who Wrote This Book?}

I'm \href{https://mcmullarkey.github.io/}{Michael Mullarkey}, a clinical
psychology PhD who solves problems where data science and product
overlap.

I've also put together \emph{a lot} of data engineering and validation
pipelines for smaller organizations. I'm a former academic who has
worked multiple types of data jobs in industry. I enjoy helping my cat
see birds in the window, watching horror movies with my spouse, and
bringing a social science lens to software engineering.

\bookmarksetup{startatroot}

\hypertarget{building-our-first-pipeline}{%
\chapter{Building Our First
Pipeline}\label{building-our-first-pipeline}}

A data pipeline has an unfortunate number of meanings. In this book
we're talking about a repeatable, scalable process for ingesting data so
that end-users can use it. Let's jump right into the building our first
pipeline.

\hypertarget{load-packages}{%
\section{Load Packages}\label{load-packages}}

\begin{Shaded}
\begin{Highlighting}[]
\FunctionTok{library}\NormalTok{(googlesheets4)}
\FunctionTok{library}\NormalTok{(readr)}
\FunctionTok{library}\NormalTok{(here)}
\end{Highlighting}
\end{Shaded}

\hypertarget{what-you-would-do-with-a-local-file}{%
\section{What You Would Do With a Local
File}\label{what-you-would-do-with-a-local-file}}

If we're doing what small organizations often do, we download a
\texttt{.csv} file from somewhere, stick it in the same folder as our
code, and run \texttt{read\_csv()} with the vague hope the data will
read in correctly.

\begin{Shaded}
\begin{Highlighting}[]
\NormalTok{penguins\_local }\OtherTok{\textless{}{-}} \FunctionTok{read\_csv}\NormalTok{(}\StringTok{"palmer\_penguins.csv"}\NormalTok{)}
\end{Highlighting}
\end{Shaded}

The aim of a data pipeline is to remove the manual steps like
downloading the file and make the hope our data is correct more grounded
in evidence.

\hypertarget{starting-a-reproducible-pipeline}{%
\section{Starting a Reproducible
Pipeline}\label{starting-a-reproducible-pipeline}}

``Reading a remote file'' can sound scary, but R can help make it as
straightforward as reading in a local file. Using the
\texttt{googlesheets4} package, we'll read in a remote file that
contains the raw
\href{https://allisonhorst.github.io/palmerpenguins/}{Palmer Penguins}
data.

\begin{Shaded}
\begin{Highlighting}[]
\CommentTok{\# Since this Google Sheet is public to anyone with the link we don\textquotesingle{}t need}
\CommentTok{\# to authenticate}

\FunctionTok{gs4\_deauth}\NormalTok{()}

\CommentTok{\# Get the URL of our Google Sheet}

\NormalTok{url }\OtherTok{\textless{}{-}} \StringTok{"https://docs.google.com/spreadsheets/d/1v0lG{-}4arxF\_zCCpfoUzCydwzaea7QqWTTQzTr8Dompw/edit?usp=sharing"}

\CommentTok{\# Read in the penguins data using the googlesheets4 package}

\NormalTok{penguins\_remote }\OtherTok{\textless{}{-}} \FunctionTok{read\_sheet}\NormalTok{(url)}
\end{Highlighting}
\end{Shaded}

And we've kicked off our pipeline! We'll focus on reading remotely from
Google Sheets in this textbook since everyone will be able to access it
for learning purposes. See this footnote for R packages that can help
you read in your data from other sources.\footnote{The
  \href{https://docs.ropensci.org/qualtRics/}{qualtRics package} for
  Qualtrics,
  \href{http://soubhikbarari.com/svmkR/articles/svmkR.html}{svmkR} for
  Survey Monkey,
  \href{https://cran.r-project.org/web/packages/rtypeform/readme/README.html}{rtypeform}
  for Typeform}

\hypertarget{only-one-more-step-to-complete-a-minimal-reproducible-pipeline}{%
\section{Only One More Step to Complete a Minimal, Reproducible
Pipeline}\label{only-one-more-step-to-complete-a-minimal-reproducible-pipeline}}

The final step of our minimal pipeline is writing the data to a location
where end-users can access it. In this case we'll write the raw Palmer
Penguins data to the ``raw'' folder in our R Project.\footnote{If you're
  not already I'd highly recommend organizing your R infrastrcture using
  Projects, see
  \href{https://r4ds.hadley.nz/workflow-scripts.html\#projects}{this
  resource} on how to get started} We use the \texttt{here} package to
make sure we don't lose track of our files.\footnote{Also note that we
  have to create the ``raw/'' folder within our project before writing
  this code, or else we'll get an ambiguous error}

\begin{Shaded}
\begin{Highlighting}[]
\NormalTok{output\_path }\OtherTok{\textless{}{-}}\NormalTok{ here}\SpecialCharTok{::}\FunctionTok{here}\NormalTok{(}\StringTok{"data\_from\_gsheets/raw/palmer\_penguins.csv"}\NormalTok{)}

\FunctionTok{write\_csv}\NormalTok{(penguins\_remote, output\_path)}
\end{Highlighting}
\end{Shaded}

And our first, minimal data pipeline is complete! We've created a
process where we could open an R Project, run a script, and have an
updated version of our data available to end-users. We're just getting
started, and in the next chapter we'll make some much needed upgrades to
this minimal pipeline.

\bookmarksetup{startatroot}

\hypertarget{initial-pipeline-upgrades}{%
\chapter{Initial Pipeline Upgrades}\label{initial-pipeline-upgrades}}

\begin{Shaded}
\begin{Highlighting}[]
\FunctionTok{library}\NormalTok{(googlesheets4)}
\FunctionTok{library}\NormalTok{(tidyverse)}
\FunctionTok{library}\NormalTok{(here)}
\FunctionTok{library}\NormalTok{(janitor)}
\FunctionTok{library}\NormalTok{(pointblank)}
\end{Highlighting}
\end{Shaded}

\hypertarget{improve-how-we-call-the-sheet}{%
\section{Improve How We Call The
Sheet}\label{improve-how-we-call-the-sheet}}

Improving our pipeline means looking for ways, big and small, to improve
our code.

We get an opportunity right off the bat! Rather than copy/pasting the
entire URL we can just take the sheet ID of of our Google Sheet. Getting
the same information with less keystrokes and opportunity for error is
always a win.

\begin{Shaded}
\begin{Highlighting}[]
\CommentTok{\# Since this Google Sheet is public to anyone with the link we don\textquotesingle{}t need}
\CommentTok{\# to authenticate}

\FunctionTok{gs4\_deauth}\NormalTok{()}

\CommentTok{\# Get the ID of our Google Sheet}

\NormalTok{sheet\_id }\OtherTok{\textless{}{-}} \StringTok{"1v0lG{-}4arxF\_zCCpfoUzCydwzaea7QqWTTQzTr8Dompw"}

\CommentTok{\# Read in the penguins data using the googlesheets4 package}

\NormalTok{penguins\_remote\_improved }\OtherTok{\textless{}{-}} \FunctionTok{read\_sheet}\NormalTok{(sheet\_id)}
\end{Highlighting}
\end{Shaded}

\hypertarget{still-read-in-raw-data}{%
\section{Still Read in Raw Data}\label{still-read-in-raw-data}}

We'll continue our pipeline as in the last chapter by writing our just
ingested data to the ``raw/'' folder in our R Project.

\begin{Shaded}
\begin{Highlighting}[]
\NormalTok{output\_path\_improved }\OtherTok{\textless{}{-}}\NormalTok{ here}\SpecialCharTok{::}\FunctionTok{here}\NormalTok{(}\StringTok{"data\_from\_gsheets/raw/palmer\_penguins\_improved.csv"}\NormalTok{)}

\FunctionTok{write\_csv}\NormalTok{(penguins\_remote\_improved, output\_path\_improved)}
\end{Highlighting}
\end{Shaded}

\hypertarget{add-some-pre-processing}{%
\section{Add Some Pre-Processing}\label{add-some-pre-processing}}

Let's start by using the \texttt{\{janitor\}} package to clean up our
column names. This will make it easier to work with the data moving
forward.

A big part of data engineering is naming things consistently and
choosing how to format your variables. Naming things sounds simple
\href{https://www.mediawiki.org/wiki/Naming_things}{but is notoriously
difficult}.

Keeping names simple, avoiding ambiguity where possible, and using a
consistent structure for names can all help. The
\href{https://cran.r-project.org/web/packages/janitor/vignettes/janitor.html}{\texttt{\{janitor\}}
package} helps us enforce a consistent structure for our variables in
one line of code.

\begin{Shaded}
\begin{Highlighting}[]
\NormalTok{penguins\_improved\_initial }\OtherTok{\textless{}{-}} 
  \FunctionTok{read\_csv}\NormalTok{(}\StringTok{"data\_from\_gsheets/raw/palmer\_penguins\_improved.csv"}\NormalTok{) }\SpecialCharTok{\%\textgreater{}\%} 
  \FunctionTok{clean\_names}\NormalTok{()}
\end{Highlighting}
\end{Shaded}

In the raw Palmer Penguins data the variable names' structure is all
over the place. ``studyName'' uses camel case, there's inconsistent
capitalization across names, special characters like ``('' that can
cause issues on some systems, and there are often spaces.\footnote{Which
  are annoying to remember and computers hate}

\begin{Shaded}
\begin{Highlighting}[]
\NormalTok{penguins\_remote\_improved }\SpecialCharTok{\%\textgreater{}\%} 
  \FunctionTok{names}\NormalTok{()}
\end{Highlighting}
\end{Shaded}

\begin{verbatim}
 [1] "studyName"           "Sample Number"       "Species"            
 [4] "Region"              "Island"              "Stage"              
 [7] "Individual ID"       "Clutch Completion"   "Date Egg"           
[10] "Culmen Length (mm)"  "Culmen Depth (mm)"   "Flipper Length (mm)"
[13] "Body Mass (g)"       "Sex"                 "Delta 15 N (o/oo)"  
[16] "Delta 13 C (o/oo)"   "Comments"           
\end{verbatim}

After running \texttt{clean\_names()} we get consistent ``snake
case''\footnote{Where each word is separated by an ``\_''} separation
for words, consistent lower-case for all characters, no special
characters like ``('' and no spaces. This standardization makes it
easier to remember, type, and reuse these variable names. Enforce a
standard like this for your data and your end-users will thank you.

\begin{Shaded}
\begin{Highlighting}[]
\NormalTok{penguins\_improved\_initial }\SpecialCharTok{\%\textgreater{}\%} 
  \FunctionTok{names}\NormalTok{()}
\end{Highlighting}
\end{Shaded}

\begin{verbatim}
 [1] "study_name"        "sample_number"     "species"          
 [4] "region"            "island"            "stage"            
 [7] "individual_id"     "clutch_completion" "date_egg"         
[10] "culmen_length_mm"  "culmen_depth_mm"   "flipper_length_mm"
[13] "body_mass_g"       "sex"               "delta_15_n_o_oo"  
[16] "delta_13_c_o_oo"   "comments"         
\end{verbatim}

\hypertarget{scanning-our-data-for-problems}{%
\section{Scanning Our Data for
Problems}\label{scanning-our-data-for-problems}}

Now that we've made our variable names easier to work with let's dive
into the actual data. The \texttt{scan\_data()} function from the
\texttt{\{pointblank\}} package in R makes it easy to get a high level
overview of our data. I want the ten-thousand foot view first, so let's
look at just the overview of our data frame.

\begin{Shaded}
\begin{Highlighting}[]
\FunctionTok{scan\_data}\NormalTok{(penguins\_improved\_initial, }\AttributeTok{sections =} \StringTok{"O"}\NormalTok{)}
\end{Highlighting}
\end{Shaded}

\begin{longtable}[]{@{}
  >{\raggedright\arraybackslash}p{(\columnwidth - 2\tabcolsep) * \real{0.5000}}
  >{\raggedright\arraybackslash}p{(\columnwidth - 2\tabcolsep) * \real{0.5000}}@{}}
\toprule()
\endhead
\begin{minipage}[t]{\linewidth}\raggedright
Columns
\end{minipage} & 17 \\
\begin{minipage}[t]{\linewidth}\raggedright
Rows
\end{minipage} & 344 \\
\begin{minipage}[t]{\linewidth}\raggedright
\texttt{NA}s
\end{minipage} & 2,365 (40.44\%) \\
\begin{minipage}[t]{\linewidth}\raggedright
Duplicate Rows
\end{minipage} & 0 \\
\bottomrule()
\end{longtable}

\hypertarget{where-did-our-data-go}{%
\section{Where Did Our Data Go??}\label{where-did-our-data-go}}

Over 40\% of our data is missing, which seems like a bad sign. The
\texttt{scan\_data()} function can help us see where this data is
missing quickly with a handy visualization.

\begin{Shaded}
\begin{Highlighting}[]
\FunctionTok{scan\_data}\NormalTok{(penguins\_improved\_initial, }\AttributeTok{sections =} \StringTok{"M"}\NormalTok{)}
\end{Highlighting}
\end{Shaded}

\href{/a}{a}",l.leadingWhitespace=3===a.firstChild.nodeType,l.tbody=!a.getElementsByTagName("tbody").length,l.htmlSerialize=!!a.getElementsByTagName("link").length,l.html5Clone="\textless:nav\textgreater"!==d.createElement("nav").cloneNode(!0).outerHTML,c.type="checkbox",c.checked=!0,b.appendChild(c),l.appendChecked=c.checked,a.innerHTML="

Okay, so we've already narrowed our problem down to a certain set of
columns where the data seems to be almost entirely missing. Maybe it's
on purpose and we're fine?

\begin{Shaded}
\begin{Highlighting}[]
\NormalTok{penguins\_remote\_improved }\SpecialCharTok{\%\textgreater{}\%} 
  \FunctionTok{clean\_names}\NormalTok{() }\SpecialCharTok{\%\textgreater{}\%} 
  \FunctionTok{select}\NormalTok{(}\FunctionTok{c}\NormalTok{(culmen\_length\_mm}\SpecialCharTok{:}\NormalTok{body\_mass\_g,delta\_15\_n\_o\_oo,delta\_13\_c\_o\_oo))}
\end{Highlighting}
\end{Shaded}

\begin{verbatim}
# A tibble: 344 x 6
   culmen_length_mm culmen_depth_mm flipper_length_mm body_mass_g
   <list>           <list>          <list>            <list>     
 1 <dbl [1]>        <dbl [1]>       <dbl [1]>         <dbl [1]>  
 2 <dbl [1]>        <dbl [1]>       <dbl [1]>         <dbl [1]>  
 3 <dbl [1]>        <dbl [1]>       <dbl [1]>         <dbl [1]>  
 4 <chr [1]>        <chr [1]>       <chr [1]>         <chr [1]>  
 5 <dbl [1]>        <dbl [1]>       <dbl [1]>         <dbl [1]>  
 6 <dbl [1]>        <dbl [1]>       <dbl [1]>         <dbl [1]>  
 7 <dbl [1]>        <dbl [1]>       <dbl [1]>         <dbl [1]>  
 8 <dbl [1]>        <dbl [1]>       <dbl [1]>         <dbl [1]>  
 9 <dbl [1]>        <dbl [1]>       <dbl [1]>         <dbl [1]>  
10 <dbl [1]>        <dbl [1]>       <dbl [1]>         <dbl [1]>  
# i 334 more rows
# i 2 more variables: delta_15_n_o_oo <list>, delta_13_c_o_oo <list>
\end{verbatim}

Not quite. Most of the ``missing'' data are hiding in list-columns, a
column type in R that can contain a whole list within each row. So the
data isn't so much ``missing'' as ``inaccessible in its current
format.''

This list-column property can be useful in certain cases, but it wasn't
what we were going for here. What do we do now?

\hypertarget{data-engineering-is-about-tradeoffs-procedures}{%
\section{Data Engineering is About Tradeoffs +
Procedures}\label{data-engineering-is-about-tradeoffs-procedures}}

One rule we've likely all heard is to never touch ``raw'' data. However,
in a data engineering role the ``raw'' data can be completely unusable,
nonsensical, or worse.

Having a reproducible pipeline with automated checks for data quality
can help, and there's
\href{https://www.thenewatlantis.com/publications/why-data-is-never-raw}{still
a human\footnote{or ideally a set of humans} making these decisions}.

Therefore, while it's good practice to not edit ``raw'' data files once
they're established we still want the ``raw'' data to contain the
information we need. Let's figure out how to rescuse the information out
of those list-columns.

Since I happen to know these values are ``nested'' in lists maybe we can
access them by using the \texttt{unnest()} function. Let's try that on
the offending columns.

\begin{Shaded}
\begin{Highlighting}[]
\NormalTok{penguins\_remote\_improved }\SpecialCharTok{\%\textgreater{}\%}
  \FunctionTok{clean\_names}\NormalTok{() }\SpecialCharTok{\%\textgreater{}\%}
  \FunctionTok{unnest}\NormalTok{(}\FunctionTok{c}\NormalTok{(culmen\_length\_mm}\SpecialCharTok{:}\NormalTok{body\_mass\_g,delta\_15\_n\_o\_oo,delta\_13\_c\_o\_oo))}
\end{Highlighting}
\end{Shaded}

\begin{verbatim}
Error in `list_unchop()`:
! Can't combine `x[[1]]` <double> and `x[[4]]` <character>.
\end{verbatim}

\hypertarget{are-we-doomed}{%
\section{Are We Doomed?}\label{are-we-doomed}}

Nope! Errors are an inevitable part of improving a pipeline. Also, if
you're not finding anything weird with your pipeline I'd be more nervous
than if you find errors like this one.

In this case, we see there are values of multiple types\footnote{type
  double and type character} in a column that can't be combined. I have
a hunch based on experience, which I can investigate further by looking
at the ``sex'' column.

\begin{Shaded}
\begin{Highlighting}[]
\NormalTok{penguins\_remote\_improved }\SpecialCharTok{\%\textgreater{}\%} 
  \FunctionTok{clean\_names}\NormalTok{() }\SpecialCharTok{\%\textgreater{}\%} 
  \FunctionTok{count}\NormalTok{(sex)}
\end{Highlighting}
\end{Shaded}

\begin{verbatim}
# A tibble: 3 x 2
  sex        n
  <chr>  <int>
1 FEMALE   165
2 MALE     168
3 NA        11
\end{verbatim}

Aha! It looks like the value ``NA'' is getting read as a character
variable rather than a missing value. This could be what's going on with
our initial read using \texttt{read\_sheet()} Let's investigate.

\begin{Shaded}
\begin{Highlighting}[]
\NormalTok{penguins\_na\_fix }\OtherTok{\textless{}{-}} \FunctionTok{read\_sheet}\NormalTok{(sheet\_id, }\AttributeTok{na =} \StringTok{"NA"}\NormalTok{)}

\NormalTok{penguins\_na\_fix }\SpecialCharTok{\%\textgreater{}\%} 
  \FunctionTok{clean\_names}\NormalTok{() }\SpecialCharTok{\%\textgreater{}\%} 
  \FunctionTok{select}\NormalTok{(}\FunctionTok{c}\NormalTok{(culmen\_length\_mm}\SpecialCharTok{:}\NormalTok{body\_mass\_g,delta\_15\_n\_o\_oo,delta\_13\_c\_o\_oo))}
\end{Highlighting}
\end{Shaded}

\begin{verbatim}
# A tibble: 344 x 6
   culmen_length_mm culmen_depth_mm flipper_length_mm body_mass_g
              <dbl>           <dbl>             <dbl>       <dbl>
 1             39.1            18.7               181        3750
 2             39.5            17.4               186        3800
 3             40.3            18                 195        3250
 4             NA              NA                  NA          NA
 5             36.7            19.3               193        3450
 6             39.3            20.6               190        3650
 7             38.9            17.8               181        3625
 8             39.2            19.6               195        4675
 9             34.1            18.1               193        3475
10             42              20.2               190        4250
# i 334 more rows
# i 2 more variables: delta_15_n_o_oo <dbl>, delta_13_c_o_oo <dbl>
\end{verbatim}

Fantastic, the information from these columns seems to be available to
us now. Let's check the overall dataframe using scan data again, this
time with the overview + missingness plot generated simultaneously.

\begin{Shaded}
\begin{Highlighting}[]
\FunctionTok{scan\_data}\NormalTok{(penguins\_na\_fix, }\AttributeTok{sections =} \StringTok{"OM"}\NormalTok{)}
\end{Highlighting}
\end{Shaded}

\begin{longtable}[]{@{}
  >{\raggedright\arraybackslash}p{(\columnwidth - 2\tabcolsep) * \real{0.5000}}
  >{\raggedright\arraybackslash}p{(\columnwidth - 2\tabcolsep) * \real{0.5000}}@{}}
\toprule()
\endhead
\begin{minipage}[t]{\linewidth}\raggedright
Columns
\end{minipage} & 17 \\
\begin{minipage}[t]{\linewidth}\raggedright
Rows
\end{minipage} & 344 \\
\begin{minipage}[t]{\linewidth}\raggedright
\texttt{NA}s
\end{minipage} & 336 (5.75\%) \\
\begin{minipage}[t]{\linewidth}\raggedright
Duplicate Rows
\end{minipage} & 0 \\
\bottomrule()
\end{longtable}

Down to \textasciitilde6\% missing data! Most of that missingness seems
concentrated in the ``comments'' column, which we can take a quick peek
at.

\begin{Shaded}
\begin{Highlighting}[]
\NormalTok{penguins\_na\_fix }\SpecialCharTok{\%\textgreater{}\%} 
  \FunctionTok{clean\_names}\NormalTok{() }\SpecialCharTok{\%\textgreater{}\%} 
  \FunctionTok{select}\NormalTok{(comments)}
\end{Highlighting}
\end{Shaded}

\begin{verbatim}
# A tibble: 344 x 1
   comments                             
   <chr>                                
 1 Not enough blood for isotopes.       
 2 <NA>                                 
 3 <NA>                                 
 4 Adult not sampled.                   
 5 <NA>                                 
 6 <NA>                                 
 7 Nest never observed with full clutch.
 8 Nest never observed with full clutch.
 9 No blood sample obtained.            
10 No blood sample obtained for sexing. 
# i 334 more rows
\end{verbatim}

Based on our investigation and the Palmer Penguins documentation this
data is looking much more like the data we'd expect.

\hypertarget{an-already-improved-pipeline}{%
\section{An Already Improved
Pipeline}\label{an-already-improved-pipeline}}

We have a much better pipeline than we did just a chapter ago!

How did I know about the \texttt{na} argument inside of
\texttt{read\_sheet()}? I looked at the documentation! Also, as you
develop better instincts you can press Ctrl + Spacebar to see what
parameters like \texttt{na} are available inside of a function like
\texttt{read\_sheet()}.

Let's read this version with our \texttt{NA} fix into the folder so we
can continue to upgrade our pipeline. We're going to overwrite the
previous, non-useful version of the file. However, we need to be careful
when building pipelines that we only overwrite files when we mean to!

\begin{Shaded}
\begin{Highlighting}[]
\NormalTok{output\_path\_nafix }\OtherTok{\textless{}{-}}\NormalTok{ here}\SpecialCharTok{::}\FunctionTok{here}\NormalTok{(}\StringTok{"data\_from\_gsheets/raw/palmer\_penguins\_improved.csv"}\NormalTok{)}

\FunctionTok{write\_csv}\NormalTok{(penguins\_na\_fix, output\_path\_nafix)}
\end{Highlighting}
\end{Shaded}

\bookmarksetup{startatroot}

\hypertarget{add-an-automated-test}{%
\chapter{Add an Automated Test}\label{add-an-automated-test}}

Let's run back through our pipeline and make some further improvements.
Yes, it will involve learning what an automated test is and why you'll
learn to love them!

\hypertarget{load-packages-1}{%
\section{Load Packages}\label{load-packages-1}}

\begin{Shaded}
\begin{Highlighting}[]
\FunctionTok{library}\NormalTok{(googlesheets4)}
\FunctionTok{library}\NormalTok{(tidyverse)}
\FunctionTok{library}\NormalTok{(here)}
\FunctionTok{library}\NormalTok{(janitor)}
\FunctionTok{library}\NormalTok{(pointblank)}
\FunctionTok{library}\NormalTok{(lubridate)}
\FunctionTok{library}\NormalTok{(glue)}
\end{Highlighting}
\end{Shaded}

\hypertarget{add-pre-processing-redux}{%
\section{Add Pre-Processing, Redux}\label{add-pre-processing-redux}}

Let's start by using the \texttt{\{janitor\}} package to clean up our
column names. This will make it easier to work with the data moving
forward.

You'll notice I had to call the \texttt{clean\_names()} function many
times in the previous chapter, which we want to avoid. Ideally we copy
and paste as little as possible while writing code.

\begin{Shaded}
\begin{Highlighting}[]
\NormalTok{penguins\_improved\_redux }\OtherTok{\textless{}{-}} 
  \FunctionTok{read\_csv}\NormalTok{(}\StringTok{"data\_from\_gsheets/raw/palmer\_penguins\_improved.csv"}\NormalTok{) }\SpecialCharTok{\%\textgreater{}\%} 
  \FunctionTok{clean\_names}\NormalTok{()}
\end{Highlighting}
\end{Shaded}

\hypertarget{stop-coding}{%
\section{Stop Coding}\label{stop-coding}}

A key part to improving our code often involves taking a step back.
Rather than charging ahead we can figure out what we want to accomplish
and the steps we need to get there.

\hypertarget{make-a-plan}{%
\section{Make a Plan}\label{make-a-plan}}

A common practice in software development is to write down a sketch of
the data we have and a sketch of the data we want. Then, we chart out
the individual steps to get us from where we are to where we want to be.

This approach can help us build stronger intuitions about how to
engineer and validate data. Using this strategy also prevents us from
getting lost in a maze of code we don't understand why we wrote.

There are plenty of ways to make this plan happen. You can write it in a
physical notebook, a Google Doc, etc. I tend to like writing pseudocode
with comments like my solutions already magically exist. For example:

\begin{Shaded}
\begin{Highlighting}[]
\CommentTok{\# I have data with duplicate IDs in it right now}

\NormalTok{non\_duplicated\_data }\OtherTok{\textless{}{-}}\NormalTok{ duplicated\_data }\SpecialCharTok{\%\textgreater{}\%} 
  \CommentTok{\# I want to remove those duplicate IDs only leaving the most recent row but retain all information}
  \FunctionTok{disapper\_duplicates}\NormalTok{(}\AttributeTok{most\_recent =} \ConstantTok{TRUE}\NormalTok{, }\AttributeTok{keep\_everything =} \ConstantTok{TRUE}\NormalTok{) }\SpecialCharTok{\%\textgreater{}\%} 
  \CommentTok{\# And then have data with no duplicates, all other info, and only the most recent of each ID}
  \FunctionTok{print}\NormalTok{()}
\end{Highlighting}
\end{Shaded}

\begin{verbatim}
Error in disapper_duplicates(., most_recent = TRUE, keep_everything = TRUE): could not find function "disapper_duplicates"
\end{verbatim}

\texttt{disappear\_duplicates()} isn't a real function, and now that
I've thought about what I'm looking to accomplish finding a real
function will be easier.

\hypertarget{our-goal-in-data-engineering-is-tidy-accurate-data}{%
\subsection{Our Goal in Data Engineering is Tidy, Accurate
Data}\label{our-goal-in-data-engineering-is-tidy-accurate-data}}

When we're making a plan while data engineering, we know we want to end
up with tidy, accurate data. Our engineering and validation efforts are
steps toward this ultimate goal.

\hypertarget{the-definition-of-tidy-depends-on-context}{%
\subsection{The Definition of Tidy Depends on
Context}\label{the-definition-of-tidy-depends-on-context}}

\begin{itemize}
\tightlist
\item
  Tidy data is
  \href{https://cran.r-project.org/web/packages/tidyr/vignettes/tidy-data.html}{data
  where}:

  \begin{itemize}
  \tightlist
  \item
    Every column is a variable
  \item
    Every row is an observation
  \item
    Every cell is a single value
  \end{itemize}
\end{itemize}

However, what counts as a ``single value'' or ``observation'' can vary
based on context. For example, if we want data with every measurement of
our penguins our defintion of ``a single value'' will be different than
if we want data with only the most recent measurement from each penguin.

Code and automated pipelines can't make these decisions for us. These
are people problems masquerading as coding problems.

\hypertarget{the-definiton-of-accurate-depends-on-context}{%
\subsection{The Definiton of Accurate Depends on
Context}\label{the-definiton-of-accurate-depends-on-context}}

This is even more true when it comes to an organization's definition of
``accurate.'' Almost no real-world dataset will ever achieve 100\%
accuracy in its data, so we have to prioritize. How much do we care if
there are duplicate IDs in the data? Impossible values in our key
outcomes?

Behaviors betray priorities here. We implicitly say we don't care much
about potential errors if we don't bother to check. Or at least we don't
care as much about those errors as other organizational priorities. That
decision might be right or wrong, but either way it's a human process
that no amount of code can solve directly.

\hypertarget{convert-plans-into-real-code}{%
\section{Convert Plans Into Real
Code}\label{convert-plans-into-real-code}}

Let's say our definition of tidy data in this case is only one
measurement per penguin. We also know\footnote{Read: Have made up for
  demonstration purposes} that we only want the last measurement
collected from each penguin. We already wrote pseudocode with comments
up above to handle this situation. We can express those plans with
actual code here:

\begin{Shaded}
\begin{Highlighting}[]
\NormalTok{penguins\_unique\_sample }\OtherTok{\textless{}{-}}\NormalTok{ penguins\_improved\_redux }\SpecialCharTok{\%\textgreater{}\%} \CommentTok{\# Data with duplicates}
  \FunctionTok{arrange}\NormalTok{(}\FunctionTok{desc}\NormalTok{(date\_egg)) }\SpecialCharTok{\%\textgreater{}\%} \CommentTok{\# Order the dataset with most recent measurements first}
  \FunctionTok{distinct}\NormalTok{(individual\_id, }\AttributeTok{.keep\_all =} \ConstantTok{TRUE}\NormalTok{) }\CommentTok{\# Keep only distinct individual\_id rows and keep all the other columns}
\end{Highlighting}
\end{Shaded}

\hypertarget{we-cant-assume-our-plans-will-work}{%
\section{We Can't Assume Our Plans Will
Work}\label{we-cant-assume-our-plans-will-work}}

If the above code is executed well we should get what we want, and we
should check behind ourselves!

We tend to do this in an ad-hoc way in our code. The most common way is
to see if the code runs at all. If it doesn't we can safely assume we
failed. The next most common is just printing out output and seeing if
it looks ok.

These are understandable approaches, and we can do better

\hypertarget{lets-automatically-check-if-our-plans-worked}{%
\section{Let's Automatically Check If Our Plans
Worked}\label{lets-automatically-check-if-our-plans-worked}}

Luckily the the \texttt{\{pointblank\}} package is way more than the
\texttt{scan\_data()} function we used in the last chapter. It's an
entire ecosystem for checking our data more thoroughly, creating
reports, and even emailing those reports automatically.

You can find more info at the package's
\href{https://rstudio.github.io/pointblank/}{docs} or this
\href{https://www.youtube.com/watch?v=hxkTbnIXI-o}{Youtube video with
demonstrations}. We can demonstrate its functionality here by checking
to make sure each \texttt{individual\_id} in our penguins data is unique
now.

\begin{Shaded}
\begin{Highlighting}[]
\CommentTok{\# Create a pointblank \textasciigrave{}agent\textasciigrave{} object, with the}
\CommentTok{\# penguins\_unique\_sample as the target table. Use one validation}
\CommentTok{\# functions, then, \textasciigrave{}interrogate()\textasciigrave{}. The agent will}
\CommentTok{\# then have some useful intel.}

\NormalTok{penguins\_distinct\_agent }\OtherTok{\textless{}{-}} 
\NormalTok{  penguins\_unique\_sample }\SpecialCharTok{\%\textgreater{}\%}
  \FunctionTok{create\_agent}\NormalTok{(}
    \AttributeTok{label =} \StringTok{"Check for unique penguin ids"}\NormalTok{,}
    \AttributeTok{actions =} \FunctionTok{action\_levels}\NormalTok{(}\AttributeTok{warn\_at =} \DecValTok{1}\NormalTok{)}
\NormalTok{  ) }\SpecialCharTok{\%\textgreater{}\%}
  \FunctionTok{rows\_distinct}\NormalTok{(}
    \FunctionTok{vars}\NormalTok{(individual\_id)}
\NormalTok{  ) }\SpecialCharTok{\%\textgreater{}\%}
  \FunctionTok{interrogate}\NormalTok{()}
\end{Highlighting}
\end{Shaded}

\hypertarget{making-the-results-of-our-automatic-tests-visible}{%
\section{Making The Results of Our Automatic Tests
Visible}\label{making-the-results-of-our-automatic-tests-visible}}

We want to be able to figure out if our tests passed or failed easily.
If something's wrong with the data, we want our process to ``fail''
loudly and in a way we can see! That loud ``failure'' is a success in
data engineering because we've prevented contaminiated data from
infecting the rest of our process.

Luckily \texttt{\{pointblank\}} gives us easy to understand reports when
we call an \texttt{agent} object like
\texttt{penguins\_distinct\_agent}. The ``warning'' circle isn't filled
in under ``W'' which means our automated test has passed!

\begin{Shaded}
\begin{Highlighting}[]
\NormalTok{penguins\_distinct\_agent}
\end{Highlighting}
\end{Shaded}

\setlength{\LTpost}{0mm}
\begin{longtable}{lrlllccrrrcccc}
\caption*{
{\large Pointblank Validation} \\ 
{\small Check for unique penguin ids}
} \\ 
\toprule
 &  & STEP & COLUMNS & VALUES & TBL & EVAL & UNITS & PASS & FAIL & W & S & N & EXT \\ 
\midrule
 & 1 &  &  & — &                                                              & ✓ & <code>$190$</code> & <code>$190$</code><br><code>$1.00$</code> & <code>$0$</code><br><code>$0.00$</code> & ○ & --- & --- & --- \\ 
\bottomrule
\end{longtable}
\begin{minipage}{\linewidth}
2023-09-27 10:40:13 MDT
\textless{} 1 s
2023-09-27 10:40:13 MDT\\
\end{minipage}

\hypertarget{we-will-never-have-100-converage-with-automatic-tests}{%
\section{We Will Never Have 100\% Converage with Automatic
Tests}\label{we-will-never-have-100-converage-with-automatic-tests}}

Which tests we create are human, scientific issues that we enshrine in
code. For example, I didn't include any kind of check here to see if we
for sure got the most recent sample from each penguin.

Since this is for demonstration purposes, that's fine. If it's crucial
for our research questions of interest that only the most recent sample
be used, we should build a check for that too.

Every decision we make is a trade-off. So while we won't get 100\%
coverage with these automatic tests, we should prioritize testing things
that would catastrophically impact our data.

\hypertarget{finish-this-upgraded-pipeline}{%
\section{Finish This Upgraded
Pipeline}\label{finish-this-upgraded-pipeline}}

We then write this processed data without duplicate IDs into a
``processed/'' folder to finish out our pipeline.

\begin{Shaded}
\begin{Highlighting}[]
\NormalTok{output\_path\_processed }\OtherTok{\textless{}{-}}\NormalTok{ here}\SpecialCharTok{::}\FunctionTok{here}\NormalTok{(}\StringTok{"data\_from\_gsheets/processed/palmer\_penguins\_improved.csv"}\NormalTok{)}

\FunctionTok{write\_csv}\NormalTok{(penguins\_unique\_sample, output\_path\_processed)}
\end{Highlighting}
\end{Shaded}

And let's also write a version of the automated test report into another
folder.

\begin{Shaded}
\begin{Highlighting}[]
\NormalTok{report\_path\_processed }\OtherTok{\textless{}{-}}\NormalTok{ here}\SpecialCharTok{::}\FunctionTok{here}\NormalTok{(}\FunctionTok{glue}\NormalTok{(}\StringTok{"reports/processed/\{now()\}\_palmer\_penguins\_report.html"}\NormalTok{))}

\CommentTok{\# Commented out here so we don\textquotesingle{}t create a new version of the report during every reload}
\CommentTok{\# export\_report(penguins\_distinct\_agent, report\_path\_processed)}
\end{Highlighting}
\end{Shaded}

If all our end-users need is penguins data without duplicate IDs we're
set! And we even have a report that will automatically save when we run
the script so we'll know if we passed the data quality checks.

And most organizations have far more than one dataset of interest. Let's
extend our pipeline to automatically handle multiple datasets.

\bookmarksetup{startatroot}

\hypertarget{handle-multiple-datasets}{%
\chapter{Handle Multiple Datasets}\label{handle-multiple-datasets}}

\begin{Shaded}
\begin{Highlighting}[]
\FunctionTok{library}\NormalTok{(googlesheets4)}
\FunctionTok{library}\NormalTok{(tidyverse)}
\FunctionTok{library}\NormalTok{(here)}
\FunctionTok{library}\NormalTok{(janitor)}
\FunctionTok{library}\NormalTok{(pointblank)}
\FunctionTok{library}\NormalTok{(glue)}
\end{Highlighting}
\end{Shaded}

\hypertarget{what-if-we-want-our-pipeline-to-include-more-than-one-dataset}{%
\section{What If We Want Our Pipeline to Include More Than One
Dataset?}\label{what-if-we-want-our-pipeline-to-include-more-than-one-dataset}}

Nearly every organization will have multiple datasets in need of
engineering. A temptation when updating our pipeline is to just
copy/paste the code we used for a previous dataset.

While this might be faster at first, this approach scales terribly. And
even if you have the finest attention to detail you're bound to make a
mistake at some point.

So what do we do instead?

We use functions that allow us to do the same process many times
automatically. These functions will allow us to write a similar amount
of code for handling 100 datasets as we did for handling 1 dataset.

The first function we'll use to read in three datasets at once is called
\texttt{map()}.

\hypertarget{modifying-our-previous-code}{%
\section{Modifying Our Previous
Code}\label{modifying-our-previous-code}}

\begin{Shaded}
\begin{Highlighting}[]
\CommentTok{\# Since this Google Sheet is public to anyone with the link we don\textquotesingle{}t need}
\CommentTok{\# to authenticate}

\FunctionTok{gs4\_deauth}\NormalTok{()}

\CommentTok{\# Get the IDs of our Google Sheets}

\NormalTok{sheet\_ids }\OtherTok{\textless{}{-}} \FunctionTok{c}\NormalTok{(}\StringTok{"1v0lG{-}4arxF\_zCCpfoUzCydwzaea7QqWTTQzTr8Dompw"}\NormalTok{,}
               \StringTok{"1wPmFajaVyWIImmvEyf6wp16mAkX1PGC5jyKUx7s7DBQ"}\NormalTok{,}
               \StringTok{"1Biy\_OhNxkaWDteBQt7AfzD89g1j2NY7mctYr50S2WVg"}\NormalTok{)}

\CommentTok{\# Read in all the data using the googlesheets4 package}

\NormalTok{all\_datasets\_raw }\OtherTok{\textless{}{-}} \FunctionTok{map}\NormalTok{(sheet\_ids, }\SpecialCharTok{\textasciitilde{}}\FunctionTok{read\_sheet}\NormalTok{(.x, }\AttributeTok{na =} \StringTok{"NA"}\NormalTok{))}
\end{Highlighting}
\end{Shaded}

There are more complicated, technically correct ways to think about what
\texttt{map()} does, but for now:

\texttt{map()} lets us write code like we're doing the process
once\footnote{\texttt{read\_sheet("1v0lG-4arxF\_zCCpfoUzCydwzaea7QqWTTQzTr8Dompw",\ na\ =\ "NA")}},
replace the single instance\footnote{in this case a single
  \texttt{sheet\_id}} with a placeholder \texttt{.x}, and run the code
on a bunch of instances\footnote{in this case all the
  \texttt{sheet\_ids}}.

This is a lot to get our brains around at first, and it's one of the
engines that makes \emph{scalable}, reproducible data pipelines
possible. For a resource focused on helping you understand
\texttt{map()} click
\href{https://jennybc.github.io/purrr-tutorial/index.html}{here}

\hypertarget{write-raw-data-all-at-once}{%
\section{Write Raw Data All at Once}\label{write-raw-data-all-at-once}}

What if we want to use \texttt{map()} with two inputs instead of one?
\texttt{map2()} has us covered!

\begin{Shaded}
\begin{Highlighting}[]
\CommentTok{\# Creating all output paths (Could still be further optimized!)}

\NormalTok{output\_paths\_improved }\OtherTok{\textless{}{-}} \FunctionTok{c}\NormalTok{(here}\SpecialCharTok{::}\FunctionTok{here}\NormalTok{(}\StringTok{"data\_from\_gsheets/raw/palmer\_penguins\_improved.csv"}\NormalTok{),}
\NormalTok{                           here}\SpecialCharTok{::}\FunctionTok{here}\NormalTok{(}\StringTok{"data\_from\_gsheets/raw/wellbeing.csv"}\NormalTok{),}
\NormalTok{                           here}\SpecialCharTok{::}\FunctionTok{here}\NormalTok{(}\StringTok{"data\_from\_gsheets/raw/telemetry.csv"}\NormalTok{))}

\CommentTok{\# Writing all the files as .csv (Note: Could use \textasciigrave{}walk\textasciigrave{} instead of \textasciigrave{}map\textasciigrave{} because this is side effect only output)}

\FunctionTok{map2}\NormalTok{(all\_datasets\_raw, output\_paths\_improved, }\SpecialCharTok{\textasciitilde{}}\FunctionTok{write\_csv}\NormalTok{(.x, .y))}
\end{Highlighting}
\end{Shaded}

And we can actually future-proof this process even more.

A mental exercise I use: Would this code be awful to type if there were
100 cases?

I definitely don't want to type out 100 full paths if I don't have to.

We'll also introduce a \texttt{for} loop, which works like
\texttt{map()} and is a more common way to do the same process lots of
times across coding languages.

You can also see why I favor \texttt{map()} when writing in R since it
can compress \textasciitilde20 lines of code into as few as 1.

\begin{Shaded}
\begin{Highlighting}[]
\CommentTok{\# Write way less per path}

\NormalTok{file\_names }\OtherTok{\textless{}{-}} \FunctionTok{c}\NormalTok{(}\StringTok{"palmer\_penguins\_improved"}\NormalTok{,}
                \StringTok{"wellbeing"}\NormalTok{,}
                \StringTok{"telemetry"}\NormalTok{)}

\CommentTok{\# Start with an empty list of file paths}

\NormalTok{output\_paths\_loop }\OtherTok{\textless{}{-}} \FunctionTok{list}\NormalTok{()}

\CommentTok{\# Create the for loop that will create a populated list of file paths}

\ControlFlowTok{for}\NormalTok{ (file\_name }\ControlFlowTok{in}\NormalTok{ file\_names) \{}
  
  \CommentTok{\# Create the string needed for the file path using \textasciigrave{}glue\textasciigrave{}}
  
\NormalTok{  path\_init }\OtherTok{\textless{}{-}} \FunctionTok{glue}\NormalTok{(}\StringTok{"data\_from\_gsheets/raw/\{file\_name\}.csv"}\NormalTok{)}
  
  \CommentTok{\# Create the path itself using \textasciigrave{}here\textasciigrave{}}
  \CommentTok{\# Note, we could do this all at once but it\textquotesingle{}s less readable}
  
\NormalTok{  path }\OtherTok{\textless{}{-}}\NormalTok{ here}\SpecialCharTok{::}\FunctionTok{here}\NormalTok{(path\_init)}
  
  \CommentTok{\# Append (or add to the end) the current path to the formerly empty list}
  
\NormalTok{  output\_paths\_loop }\OtherTok{\textless{}{-}} \FunctionTok{append}\NormalTok{(output\_paths\_loop, path)}
  
\NormalTok{\}}
\end{Highlighting}
\end{Shaded}

If we were following this process we'd then make sure to write our
datasets into the raw folder.

\begin{Shaded}
\begin{Highlighting}[]
\CommentTok{\# Writing all the files as .csv (Note: Could use \textasciigrave{}walk\textasciigrave{} instead of \textasciigrave{}map\textasciigrave{} because this is side effect only output)}

\FunctionTok{map2}\NormalTok{(all\_datasets\_raw, output\_paths\_loop, }\SpecialCharTok{\textasciitilde{}}\FunctionTok{write\_csv}\NormalTok{(.x, .y))}
\end{Highlighting}
\end{Shaded}

\hypertarget{add-pre-processing-to-our-multi-dataset-pipeline}{%
\section{Add Pre-Processing to Our Multi-Dataset
Pipeline}\label{add-pre-processing-to-our-multi-dataset-pipeline}}

Let's start by using the \texttt{\{janitor\}} package to clean up our
column names across all the datasets, rather than just one. We'll switch
back to \texttt{map()} for this version of ``do multiple datasets at
once.''

\begin{Shaded}
\begin{Highlighting}[]
\NormalTok{all\_datasets\_initial }\OtherTok{\textless{}{-}} \FunctionTok{map}\NormalTok{(file\_names, }\SpecialCharTok{\textasciitilde{}}\NormalTok{\{}
  
  \CommentTok{\# Create file name}
  
\NormalTok{  path\_for\_read }\OtherTok{\textless{}{-}} \FunctionTok{glue}\NormalTok{(}\StringTok{"data\_from\_gsheets/raw/\{.x\}.csv"}\NormalTok{)}
  
  \CommentTok{\# Read in the csv and clean the column names}
  
  \FunctionTok{read\_csv}\NormalTok{(path\_for\_read) }\SpecialCharTok{\%\textgreater{}\%}
    \FunctionTok{clean\_names}\NormalTok{()}
  
\NormalTok{\})}
\end{Highlighting}
\end{Shaded}

\hypertarget{lets-scan-our-data-for-any-obvious-problems}{%
\section{Let's Scan Our Data for Any Obvious
Problems}\label{lets-scan-our-data-for-any-obvious-problems}}

We can still scan a single dataset by calling its position in our list.

\begin{Shaded}
\begin{Highlighting}[]
\FunctionTok{scan\_data}\NormalTok{(all\_datasets\_initial[[}\DecValTok{2}\NormalTok{]], }\AttributeTok{sections =} \StringTok{"O"}\NormalTok{)}
\end{Highlighting}
\end{Shaded}

\begin{longtable}[]{@{}
  >{\raggedright\arraybackslash}p{(\columnwidth - 2\tabcolsep) * \real{0.5000}}
  >{\raggedright\arraybackslash}p{(\columnwidth - 2\tabcolsep) * \real{0.5000}}@{}}
\toprule()
\endhead
\begin{minipage}[t]{\linewidth}\raggedright
Columns
\end{minipage} & 11 \\
\begin{minipage}[t]{\linewidth}\raggedright
Rows
\end{minipage} & 238 \\
\begin{minipage}[t]{\linewidth}\raggedright
\texttt{NA}s
\end{minipage} & 0 \\
\begin{minipage}[t]{\linewidth}\raggedright
Duplicate Rows
\end{minipage} & 0 \\
\bottomrule()
\end{longtable}

But we can also scan all our datasets at once and then output the
reports.

\begin{Shaded}
\begin{Highlighting}[]
\NormalTok{many\_reports }\OtherTok{\textless{}{-}} \FunctionTok{map}\NormalTok{(all\_datasets\_initial, }\SpecialCharTok{\textasciitilde{}}\FunctionTok{scan\_data}\NormalTok{(.x, }\AttributeTok{sections =} \StringTok{"O"}\NormalTok{))}
\NormalTok{many\_reports[[}\DecValTok{2}\NormalTok{]]}
\end{Highlighting}
\end{Shaded}

\begin{longtable}[]{@{}
  >{\raggedright\arraybackslash}p{(\columnwidth - 2\tabcolsep) * \real{0.5000}}
  >{\raggedright\arraybackslash}p{(\columnwidth - 2\tabcolsep) * \real{0.5000}}@{}}
\toprule()
\endhead
\begin{minipage}[t]{\linewidth}\raggedright
Columns
\end{minipage} & 11 \\
\begin{minipage}[t]{\linewidth}\raggedright
Rows
\end{minipage} & 238 \\
\begin{minipage}[t]{\linewidth}\raggedright
\texttt{NA}s
\end{minipage} & 0 \\
\begin{minipage}[t]{\linewidth}\raggedright
Duplicate Rows
\end{minipage} & 0 \\
\bottomrule()
\end{longtable}

\hypertarget{we-can-automatically-check-our-data-across-many-datasets}{%
\section{We Can Automatically Check Our Data Across Many
Datasets}\label{we-can-automatically-check-our-data-across-many-datasets}}

There are certain data quality checks we're going to want to do across
all of our datasets. A lack of duplicate IDs is an example of one such
check, so let's start there. Don't worry if you don't understand all
this code yet, we're going to dive deep in the next chapter into how it
works step by step.

\begin{Shaded}
\begin{Highlighting}[]
\CommentTok{\# Need to input the datasets and the name of the id variable for each}

\NormalTok{id\_var\_names }\OtherTok{\textless{}{-}} \FunctionTok{c}\NormalTok{(}\StringTok{"individual\_id"}\NormalTok{,}\StringTok{"participant\_id"}\NormalTok{, }\StringTok{"id"}\NormalTok{)}

\NormalTok{all\_agents }\OtherTok{\textless{}{-}} \FunctionTok{map2}\NormalTok{(all\_datasets\_initial, id\_var\_names, }\SpecialCharTok{\textasciitilde{}}\NormalTok{\{}
  
\NormalTok{  .x }\SpecialCharTok{\%\textgreater{}\%}
    \FunctionTok{create\_agent}\NormalTok{(}
      \AttributeTok{label =} \StringTok{"Check for unique ids"}\NormalTok{,}
      \AttributeTok{actions =} \FunctionTok{action\_levels}\NormalTok{(}\AttributeTok{warn\_at =} \DecValTok{1}\NormalTok{)}
\NormalTok{    ) }\SpecialCharTok{\%\textgreater{}\%}
    \FunctionTok{rows\_distinct}\NormalTok{(}
      \AttributeTok{columns =} \FunctionTok{vars}\NormalTok{(}\SpecialCharTok{!!!}\NormalTok{.y)}
\NormalTok{    ) }\SpecialCharTok{\%\textgreater{}\%}
    \FunctionTok{interrogate}\NormalTok{()}
  
\NormalTok{\})}

\NormalTok{all\_agents[[}\DecValTok{3}\NormalTok{]]}
\end{Highlighting}
\end{Shaded}

\setlength{\LTpost}{0mm}
\begin{longtable}{lrlllccrrrcccc}
\caption*{
{\large Pointblank Validation} \\ 
{\small Check for unique ids}
} \\ 
\toprule
 &  & STEP & COLUMNS & VALUES & TBL & EVAL & UNITS & PASS & FAIL & W & S & N & EXT \\ 
\midrule
 & 1 &  &  & — &                                                              & ✓ & <code>$1K$</code> & <code>$1K$</code><br><code>$1.00$</code> & <code>$0$</code><br><code>$0.00$</code> & ○ & --- & --- & --- \\ 
\bottomrule
\end{longtable}
\begin{minipage}{\linewidth}
2023-09-27 10:40:28 MDT
\textless{} 1 s
2023-09-27 10:40:28 MDT\\
\end{minipage}

\hypertarget{we-now-have-a-reproducible-scalable-pipeline-with-automatic-checks-for-duplicate-ids}{%
\section{We Now Have a Reproducible, Scalable Pipeline with Automatic
Checks for Duplicate
IDs}\label{we-now-have-a-reproducible-scalable-pipeline-with-automatic-checks-for-duplicate-ids}}

It's worthwhile to pause here to think how much better than the status
quo this is at many organizations. You can run this script anytime you
want, get updated data across as many datasets as you have, and check
all those datasets for duplicate IDs. This is a great accomplishment by
itself, and we're not done yet.

We're going to tackle how these map functions work in detail and use
them to apply specific checks to some datasets but not others.

\bookmarksetup{startatroot}

\hypertarget{different-checks-for-different-datasets}{%
\chapter{Different Checks for Different
Datasets}\label{different-checks-for-different-datasets}}

\begin{Shaded}
\begin{Highlighting}[]
\FunctionTok{library}\NormalTok{(googlesheets4)}
\FunctionTok{library}\NormalTok{(tidyverse)}
\FunctionTok{library}\NormalTok{(here)}
\FunctionTok{library}\NormalTok{(janitor)}
\FunctionTok{library}\NormalTok{(pointblank)}
\FunctionTok{library}\NormalTok{(glue)}
\FunctionTok{library}\NormalTok{(gt)}
\FunctionTok{library}\NormalTok{(knitr)}
\FunctionTok{library}\NormalTok{(lubridate)}
\end{Highlighting}
\end{Shaded}

\hypertarget{lets-be-real-different-datasets-will-require-different-checks}{%
\section{Let's Be Real, Different Datasets Will Require Different
Checks}\label{lets-be-real-different-datasets-will-require-different-checks}}

While there are certain checks we'll want to do in a vast majority of
datasets,\footnote{Like check for duplicate IDs} we'll need different
automated checks for different datasets.

This might feel like a huge wrench in our plan for automated checks, but
it's not! Let's walk through how we can keep our pipeline moving.

\hypertarget{lets-read-in-all-our-raw-data}{%
\section{Let's Read in All Our Raw
Data}\label{lets-read-in-all-our-raw-data}}

\begin{Shaded}
\begin{Highlighting}[]
\CommentTok{\# Write way less per path}

\NormalTok{file\_names }\OtherTok{\textless{}{-}} \FunctionTok{c}\NormalTok{(}\StringTok{"palmer\_penguins\_improved"}\NormalTok{,}
                \StringTok{"wellbeing"}\NormalTok{,}
                \StringTok{"telemetry"}\NormalTok{)}

\NormalTok{all\_datasets\_different }\OtherTok{\textless{}{-}} \FunctionTok{map}\NormalTok{(file\_names, }\SpecialCharTok{\textasciitilde{}}\NormalTok{\{}
  
  \CommentTok{\# Create file name}
  
\NormalTok{  path\_for\_read }\OtherTok{\textless{}{-}} \FunctionTok{glue}\NormalTok{(}\StringTok{"data\_from\_gsheets/raw/\{.x\}.csv"}\NormalTok{)}
  
  \CommentTok{\# Read in the csv and clean the column names}
  
  \FunctionTok{read\_csv}\NormalTok{(path\_for\_read) }\SpecialCharTok{\%\textgreater{}\%}
    \FunctionTok{clean\_names}\NormalTok{()}
  
\NormalTok{\})}
\end{Highlighting}
\end{Shaded}

\hypertarget{creating-flags-for-different-kinds-of-checks}{%
\section{Creating `Flags' For Different Kinds of
Checks}\label{creating-flags-for-different-kinds-of-checks}}

Let's start with the example we've seen before, checking for duplicate
IDs. Let's say we're not concerned about duplicate IDs in the
\texttt{telemetry} data, but we still want to do that check in the
\texttt{palmer\_penguins} and \texttt{well\_being} datasets.

We can create a ``flag'' for the duplicate ID check. ``Flag'' is just a
somewhat fancy way of saying a variable that returns \texttt{TRUE} when
we want to perform the check and \texttt{FALSE} when we don't.

Here's what that looks like in code. Don't worry if there's a lot here
you don't understand yet, we'll go through a version step by step later!

\begin{Shaded}
\begin{Highlighting}[]
\CommentTok{\# Need to input the datasets and the name of the id variable for each}

\NormalTok{id\_var\_names }\OtherTok{\textless{}{-}} \FunctionTok{c}\NormalTok{(}\StringTok{"individual\_id"}\NormalTok{,}\StringTok{"participant\_id"}\NormalTok{, }\StringTok{"id"}\NormalTok{)}

\NormalTok{check\_dupe\_ids }\OtherTok{\textless{}{-}} \FunctionTok{c}\NormalTok{(}\FunctionTok{rep}\NormalTok{(}\ConstantTok{TRUE}\NormalTok{, }\DecValTok{2}\NormalTok{),}\FunctionTok{rep}\NormalTok{(}\ConstantTok{FALSE}\NormalTok{, }\DecValTok{1}\NormalTok{))}

\NormalTok{all\_agents\_conditional }\OtherTok{\textless{}{-}} \FunctionTok{pmap}\NormalTok{(}\FunctionTok{list}\NormalTok{(all\_datasets\_different, }
\NormalTok{                                    id\_var\_names, }
\NormalTok{                                    check\_dupe\_ids), }\SpecialCharTok{\textasciitilde{}}\NormalTok{\{}
  
  \ControlFlowTok{if}\NormalTok{(..}\DecValTok{3} \SpecialCharTok{==} \ConstantTok{TRUE}\NormalTok{) \{}
    
\NormalTok{    ..}\DecValTok{1} \SpecialCharTok{\%\textgreater{}\%}
    \FunctionTok{create\_agent}\NormalTok{(}
      \AttributeTok{label =} \StringTok{"Check for unique ids"}\NormalTok{,}
      \AttributeTok{actions =} \FunctionTok{action\_levels}\NormalTok{(}\AttributeTok{warn\_at =} \DecValTok{1}\NormalTok{)}
\NormalTok{    ) }\SpecialCharTok{\%\textgreater{}\%}
    \FunctionTok{rows\_distinct}\NormalTok{(}
      \AttributeTok{columns =} \FunctionTok{vars}\NormalTok{(}\SpecialCharTok{!!!}\NormalTok{..}\DecValTok{2}\NormalTok{)}
\NormalTok{    ) }\SpecialCharTok{\%\textgreater{}\%}
    \FunctionTok{interrogate}\NormalTok{()}
    
\NormalTok{  \} }\ControlFlowTok{else}\NormalTok{ \{}
    
    \FunctionTok{data.frame}\NormalTok{(}\AttributeTok{no\_check =} \ConstantTok{TRUE}\NormalTok{) }\SpecialCharTok{\%\textgreater{}\%} 
    \FunctionTok{gt}\NormalTok{() }\SpecialCharTok{\%\textgreater{}\%} 
      \FunctionTok{tab\_header}\NormalTok{(}\StringTok{"No Check for Duplicate IDs"}\NormalTok{)}
    
\NormalTok{  \}}
  
  
\NormalTok{\})}

\NormalTok{all\_agents\_conditional[[}\DecValTok{3}\NormalTok{]]}
\end{Highlighting}
\end{Shaded}

\begin{longtable}{c}
\caption*{
{\large No Check for Duplicate IDs}
} \\ 
\toprule
no\_check \\ 
\midrule
TRUE \\ 
\bottomrule
\end{longtable}

We have to introduce a new extension of the \texttt{map()} function:
\texttt{pmap()}.

Remember how we had \texttt{map2()} when we had two variables we wanted
to iterate over at the same time? Rather than have \texttt{map3()} for 3
inputs, \texttt{map4()} for 4 inputs, etc. we can use pmap with however
many inputs we want.

The biggest differences when using pmap is that instead of\texttt{.x}
and \texttt{.y} we could use \texttt{..1} for the first input,
\texttt{..2} for the second input, etc.

We then use an \texttt{if\ else} statement to only perform the check for
duplicate IDs when our flag equals \texttt{TRUE}. If it's false we
create a table output explicitly saying the check wasn't performed.

\hypertarget{creating-more-scalable-automated-checks}{%
\section{Creating More Scalable, Automated
Checks}\label{creating-more-scalable-automated-checks}}

This is a great start, and I could imagine this getting unwieldy with
even one more kind of check. If we wanted to add a variety of checks
that do and don't apply to different datasets this code would become an
impossible mess.

Let's add another check and create a more scalable version of this code.

First let's create the values we'll need for our checks. We'll need to
grab the names of the columns we're checking.

\begin{Shaded}
\begin{Highlighting}[]
\CommentTok{\# Create necessary input for id variable names}

\NormalTok{id\_var\_names }\OtherTok{\textless{}{-}} \FunctionTok{c}\NormalTok{(}\StringTok{"individual\_id"}\NormalTok{,}\StringTok{"participant\_id"}\NormalTok{, }\StringTok{"id"}\NormalTok{)}

\CommentTok{\# Get the names of variables whose values we want to check}

\NormalTok{vars\_value\_check }\OtherTok{\textless{}{-}} \FunctionTok{list}\NormalTok{(all\_datasets\_different[[}\DecValTok{1}\NormalTok{]] }\SpecialCharTok{\%\textgreater{}\%} 
                          \FunctionTok{select}\NormalTok{(}\FunctionTok{starts\_with}\NormalTok{(}\StringTok{"culmen"}\NormalTok{)) }\SpecialCharTok{\%\textgreater{}\%} 
                          \FunctionTok{names}\NormalTok{(),}
\NormalTok{                      all\_datasets\_different[[}\DecValTok{2}\NormalTok{]] }\SpecialCharTok{\%\textgreater{}\%} 
                        \FunctionTok{select}\NormalTok{(}\FunctionTok{starts\_with}\NormalTok{(}\StringTok{"wellbeing"}\NormalTok{)) }\SpecialCharTok{\%\textgreater{}\%} 
                        \FunctionTok{names}\NormalTok{(),}
\NormalTok{                      all\_datasets\_different[[}\DecValTok{3}\NormalTok{]] }\SpecialCharTok{\%\textgreater{}\%} 
                        \FunctionTok{select}\NormalTok{(}\FunctionTok{starts\_with}\NormalTok{(}\StringTok{"accelerometer"}\NormalTok{)) }\SpecialCharTok{\%\textgreater{}\%} 
                        \FunctionTok{names}\NormalTok{())}
\end{Highlighting}
\end{Shaded}

But we know how to spot non-scalable copy-paste + coding now! Let's
create a couple functions and map over them to get those column names
more efficiently.

\begin{Shaded}
\begin{Highlighting}[]
\NormalTok{id\_var\_names }\OtherTok{\textless{}{-}} \FunctionTok{map}\NormalTok{(all\_datasets\_different, }\SpecialCharTok{\textasciitilde{}}\NormalTok{\{}
  
\NormalTok{  .x }\SpecialCharTok{\%\textgreater{}\%} 
    \FunctionTok{select}\NormalTok{(}\FunctionTok{ends\_with}\NormalTok{(}\StringTok{"id"}\NormalTok{)) }\SpecialCharTok{\%\textgreater{}\%} 
    \FunctionTok{names}\NormalTok{()}
  
\NormalTok{\})}

\NormalTok{var\_start\_with }\OtherTok{\textless{}{-}} \FunctionTok{list}\NormalTok{(}\StringTok{"culmen"}\NormalTok{,}\StringTok{"wellbeing"}\NormalTok{,}\StringTok{"accelerometer"}\NormalTok{)}

\NormalTok{vars\_value\_check }\OtherTok{\textless{}{-}} \FunctionTok{map2}\NormalTok{(all\_datasets\_different, var\_start\_with, }\SpecialCharTok{\textasciitilde{}}\NormalTok{\{}
  
\NormalTok{  .x }\SpecialCharTok{\%\textgreater{}\%} 
    \FunctionTok{select}\NormalTok{(}\FunctionTok{starts\_with}\NormalTok{(.y)) }\SpecialCharTok{\%\textgreater{}\%} 
    \FunctionTok{names}\NormalTok{()}
  
\NormalTok{\})}
\end{Highlighting}
\end{Shaded}

Notice how the \texttt{ends\_with("id")} call is only possible because
all our ID variables end with the characters ``id''

Naming things well and consistently is just as much a part of data
engineering as writing code part 87!

Anyway, let's create the flags for our checks. We'll use the
\texttt{rep()} function to reduce keystrokes and make our flags more
scalable.

\begin{Shaded}
\begin{Highlighting}[]
\CommentTok{\# Create check flags}

\NormalTok{check\_dupe\_ids }\OtherTok{\textless{}{-}} \FunctionTok{c}\NormalTok{(}\FunctionTok{rep}\NormalTok{(}\ConstantTok{TRUE}\NormalTok{, }\DecValTok{2}\NormalTok{),}\FunctionTok{rep}\NormalTok{(}\ConstantTok{FALSE}\NormalTok{, }\DecValTok{1}\NormalTok{))}

\NormalTok{check\_var\_range }\OtherTok{\textless{}{-}} \FunctionTok{c}\NormalTok{(}\FunctionTok{rep}\NormalTok{(}\ConstantTok{FALSE}\NormalTok{, }\DecValTok{1}\NormalTok{), }\FunctionTok{rep}\NormalTok{(}\ConstantTok{TRUE}\NormalTok{, }\DecValTok{2}\NormalTok{))}
\end{Highlighting}
\end{Shaded}

And now we get to \texttt{pmap()}. First we'll create a function that
uses the \texttt{..1}, \texttt{..2}, \texttt{..3} etc. to make our
function work.

\begin{Shaded}
\begin{Highlighting}[]
\CommentTok{\# Create more scalable pmap}

\NormalTok{all\_agents\_conditional }\OtherTok{\textless{}{-}} \FunctionTok{pmap}\NormalTok{(}\AttributeTok{.l =} \FunctionTok{list}\NormalTok{(all\_datasets\_different, }
\NormalTok{                                         id\_var\_names, }
\NormalTok{                                         check\_dupe\_ids,}
\NormalTok{                                         check\_var\_range,}
\NormalTok{                                         vars\_value\_check),}
                               \SpecialCharTok{\textasciitilde{}}\NormalTok{\{}
    
\NormalTok{    ..}\DecValTok{1} \SpecialCharTok{\%\textgreater{}\%}
      \FunctionTok{create\_agent}\NormalTok{(}
        \AttributeTok{label =} \StringTok{"Conduct automated checks"}\NormalTok{,}
        \AttributeTok{actions =} \FunctionTok{action\_levels}\NormalTok{(}\AttributeTok{warn\_at =} \DecValTok{1}\NormalTok{)}
\NormalTok{      ) }\SpecialCharTok{\%\textgreater{}\%}
\NormalTok{      \{}\ControlFlowTok{if}\NormalTok{ (..}\DecValTok{3} \SpecialCharTok{==} \ConstantTok{TRUE}\NormalTok{)}
        \FunctionTok{rows\_distinct}\NormalTok{(.,}
          \AttributeTok{columns =} \FunctionTok{vars}\NormalTok{(}\SpecialCharTok{!!!}\NormalTok{..}\DecValTok{2}\NormalTok{)}
\NormalTok{          )}
        \ControlFlowTok{else}\NormalTok{ .}
\NormalTok{        \} }\SpecialCharTok{\%\textgreater{}\%}
\NormalTok{      \{}\ControlFlowTok{if}\NormalTok{ (..}\DecValTok{4} \SpecialCharTok{==} \ConstantTok{TRUE}\NormalTok{)}
\NormalTok{        pointblank}\SpecialCharTok{::}\FunctionTok{col\_vals\_between}\NormalTok{(.,}
          \AttributeTok{columns =}\NormalTok{ ..}\DecValTok{5}\NormalTok{,}
          \AttributeTok{left =} \SpecialCharTok{{-}}\DecValTok{4}\NormalTok{,}
          \AttributeTok{right =} \DecValTok{4}\NormalTok{,}
          \AttributeTok{na\_pass =} \ConstantTok{TRUE}
\NormalTok{        )}
        \ControlFlowTok{else}\NormalTok{ .}
\NormalTok{        \} }\SpecialCharTok{\%\textgreater{}\%} 
      \FunctionTok{interrogate}\NormalTok{()}

  
\NormalTok{\})}
\end{Highlighting}
\end{Shaded}

This code will technically run, give us the right output, and use
functions to make it more scalable.

It's also really hard to figure out which variable is which and what is
happening! Let's create a version where we name all the variables we're
passing into \texttt{pmap()} instead.

\begin{Shaded}
\begin{Highlighting}[]
\NormalTok{all\_agents\_conditional }\OtherTok{\textless{}{-}} \FunctionTok{pmap}\NormalTok{(}\AttributeTok{.l =} \FunctionTok{list}\NormalTok{(}\AttributeTok{dataset\_for\_check =}\NormalTok{ all\_datasets\_different, }
                                         \AttributeTok{id =}\NormalTok{ id\_var\_names, }
                                         \AttributeTok{check\_dupe\_ids =}\NormalTok{ check\_dupe\_ids,}
                                         \AttributeTok{check\_var\_range =}\NormalTok{ check\_var\_range,}
                                         \AttributeTok{vars\_value\_check =}\NormalTok{ vars\_value\_check),}
                               \AttributeTok{.f =} \ControlFlowTok{function}\NormalTok{(dataset\_for\_check,}
\NormalTok{                                             id,}
\NormalTok{                                             check\_dupe\_ids,}
\NormalTok{                                             check\_var\_range,}
\NormalTok{                                             vars\_value\_check) \{}
    
\NormalTok{    dataset\_for\_check }\SpecialCharTok{\%\textgreater{}\%}
      \FunctionTok{create\_agent}\NormalTok{(}
        \AttributeTok{label =} \StringTok{"Conduct automated tests"}\NormalTok{,}
        \AttributeTok{actions =} \FunctionTok{action\_levels}\NormalTok{(}\AttributeTok{warn\_at =} \DecValTok{1}\NormalTok{)}
\NormalTok{      ) }\SpecialCharTok{\%\textgreater{}\%}
\NormalTok{      \{}\ControlFlowTok{if}\NormalTok{ (check\_dupe\_ids }\SpecialCharTok{==} \ConstantTok{TRUE}\NormalTok{)}
        \FunctionTok{rows\_distinct}\NormalTok{(.,}
          \AttributeTok{columns =} \FunctionTok{vars}\NormalTok{(}\SpecialCharTok{!!!}\NormalTok{id)}
\NormalTok{          )}
        \ControlFlowTok{else}\NormalTok{ .}
\NormalTok{        \} }\SpecialCharTok{\%\textgreater{}\%}
\NormalTok{      \{}\ControlFlowTok{if}\NormalTok{ (check\_var\_range }\SpecialCharTok{==} \ConstantTok{TRUE}\NormalTok{)}
\NormalTok{        pointblank}\SpecialCharTok{::}\FunctionTok{col\_vals\_between}\NormalTok{(.,}
          \AttributeTok{columns =}\NormalTok{ vars\_value\_check,}
          \AttributeTok{left =} \SpecialCharTok{{-}}\DecValTok{4}\NormalTok{,}
          \AttributeTok{right =} \DecValTok{4}\NormalTok{,}
          \AttributeTok{na\_pass =} \ConstantTok{TRUE}
\NormalTok{        )}
        \ControlFlowTok{else}\NormalTok{ .}
\NormalTok{        \} }\SpecialCharTok{\%\textgreater{}\%} 
      \FunctionTok{interrogate}\NormalTok{()}

  
\NormalTok{\})}
\end{Highlighting}
\end{Shaded}

\hypertarget{a-deep-dive-into-pmap}{%
\section{\texorpdfstring{A Deep Dive Into
\texttt{pmap()}}{A Deep Dive Into pmap()}}\label{a-deep-dive-into-pmap}}

Still probably intimidating if you've never seen a \texttt{pmap()}
function before.\footnote{Or even if you have!} Let's break down each
part of this function using plain language.

This part of the function lets us name our inputs. We can either give
them shorter names (e.g., ``id'' instead of ``id\_var\_names'') or keep
them the same (``check\_var\_range'').

\begin{Shaded}
\begin{Highlighting}[]
\CommentTok{\# .l = list(dataset\_for\_check = all\_datasets\_different,}
\CommentTok{\#           id = id\_var\_names,}
\CommentTok{\#           check\_dupe\_ids = check\_dupe\_ids, }
\CommentTok{\#           check\_var\_range = check\_var\_range,}
\CommentTok{\#           vars\_value\_check = vars\_value\_check)}
\end{Highlighting}
\end{Shaded}

This part of the function let's \texttt{pmap} know we're going to
execute a function using the parameters we just created instead of
\texttt{..l} etc. Naming these parameters here lets us use those names
in the function itself, which is way easier to understand than
\texttt{..1} or \texttt{..2}.

\begin{Shaded}
\begin{Highlighting}[]
\CommentTok{\# .f = function(dataset\_for\_check,}
\CommentTok{\#               id,}
\CommentTok{\#               check\_dupe\_ids,}
\CommentTok{\#               check\_var\_range,}
\CommentTok{\#               vars\_value\_check}
\CommentTok{\#               )}
\end{Highlighting}
\end{Shaded}

In this chunk of the code we take a dataset from our list of raw
datasets and create an agent using the \texttt{\{pointblank\}} package.
We also tell the process to warn us if there's even one
failure.\footnote{Though we don't stop the process at all becuase we
  want all the checks to run even if an early one ``fails''}

\begin{Shaded}
\begin{Highlighting}[]
\CommentTok{\# dataset\_for\_check \%\textgreater{}\%}
\CommentTok{\#       create\_agent(}
\CommentTok{\#         label = "Conduct automated tests",}
\CommentTok{\#         actions = action\_levels(warn\_at = 1)}
\CommentTok{\#       ) }
\end{Highlighting}
\end{Shaded}

This looks a bit different than anything we've seen so far. What this
code allows us to do is say ``Hey, if we said we wanted to check if
there were duplicate IDs\footnote{\texttt{if\ (check\_dupe\_ids\ ==\ TRUE)}}
then check if there are duplicate IDs\footnote{\texttt{rows\_distinct(.,columns\ =\ vars(!!!id))}}.
Otherwise, just pass the dataframe through the pipleline as is without
checking for duplicate IDs{[}\texttt{else\ .}{]}.''

Another note, the \texttt{!!!id} looks a bit wild, and you don't have to
fully understand what's happening there right now. All you need to know
is this is one way to ``unquote'' variables that are passed in as
character values\footnote{like ``individual\_id''} but for coding
purposes need to not have quotes around them.\footnote{individual\_id}
For more in depth discussion of this topic you can check out
\href{https://dplyr.tidyverse.org/articles/programming.html}{this
resource}.

\begin{Shaded}
\begin{Highlighting}[]
\CommentTok{\# \{if (check\_dupe\_ids == TRUE)}
\CommentTok{\#         rows\_distinct(.,}
\CommentTok{\#           columns = vars(!!!id)}
\CommentTok{\#           )}
\CommentTok{\#         else .}
\CommentTok{\#         \}}
\end{Highlighting}
\end{Shaded}

This chunk follows the same logic as the previous chunk, just for a
different kind of check. ``Hey, if we said we wanted to check if some
columns only had values inside of a certain range\footnote{\texttt{if\ (check\_var\_range\ ==\ TRUE)}}
then check the columns we specify for only values between a certain
range\footnote{\texttt{pointblank::col\_vals\_between(.,columns\ =\ vars\_value\_check,left\ =\ -4,right\ =\ 4,na\_pass\ =\ TRUE)}}.
Otherwise, just pass the dataframe through the pipleline as is without
checking for if some columns only had values inside of a certain
range{[}\texttt{else\ .}{]}.''

\begin{Shaded}
\begin{Highlighting}[]
\CommentTok{\# \{if (check\_var\_range == TRUE)}
\CommentTok{\#         pointblank::col\_vals\_between(.,}
\CommentTok{\#           columns = vars\_value\_check,}
\CommentTok{\#           left = {-}4,}
\CommentTok{\#           right = 4,}
\CommentTok{\#           na\_pass = TRUE}
\CommentTok{\#         )}
\CommentTok{\#         else .}
\CommentTok{\#         \}}
\end{Highlighting}
\end{Shaded}

And this final bit produces the reports for each dataset!

\begin{Shaded}
\begin{Highlighting}[]
\CommentTok{\#interrogate()}
\end{Highlighting}
\end{Shaded}

\hypertarget{looking-at-our-reports}{%
\section{Looking at Our Reports}\label{looking-at-our-reports}}

Lets look at those reports now.

\begin{Shaded}
\begin{Highlighting}[]
\CommentTok{\# Breaking my own rule about not copy/pasting code here since if I try }
\CommentTok{\# to print list of agents all at once the formatting looks terrible }
\CommentTok{\# in the book}

\NormalTok{all\_agents\_conditional[[}\DecValTok{1}\NormalTok{]]}
\end{Highlighting}
\end{Shaded}

\setlength{\LTpost}{0mm}
\begin{longtable}{lrlllccrrrcccc}
\caption*{
{\large Pointblank Validation} \\ 
{\small Conduct automated tests}
} \\ 
\toprule
 &  & STEP & COLUMNS & VALUES & TBL & EVAL & UNITS & PASS & FAIL & W & S & N & EXT \\ 
\midrule
 & 1 &  &  & — &                                                              & ✓ & <code>$344$</code> & <code>$76$</code><br><code>$0.22$</code> & <code>$268$</code><br><code>$0.78$</code> & ● & --- & --- &  \\ 
\bottomrule
\end{longtable}
\begin{minipage}{\linewidth}
2023-09-27 10:40:43 MDT
\textless{} 1 s
2023-09-27 10:40:43 MDT\\
\end{minipage}

\begin{Shaded}
\begin{Highlighting}[]
\NormalTok{all\_agents\_conditional[[}\DecValTok{2}\NormalTok{]]}
\end{Highlighting}
\end{Shaded}

\setlength{\LTpost}{0mm}
\begin{longtable}{lrlllccrrrcccc}
\caption*{
{\large Pointblank Validation} \\ 
{\small Conduct automated tests}
} \\ 
\toprule
 &  & STEP & COLUMNS & VALUES & TBL & EVAL & UNITS & PASS & FAIL & W & S & N & EXT \\ 
\midrule
 & 1 &  &  & — &                                                              & ✓ & <code>$238$</code> & <code>$234$</code><br><code>$0.98$</code> & <code>$4$</code><br><code>$0.02$</code> & ● & --- & --- &  \\ 
 & 2 &  &  &  &                                                              & ✓ & <code>$238$</code> & <code>$238$</code><br><code>$1.00$</code> & <code>$0$</code><br><code>$0.00$</code> & ○ & --- & --- & --- \\ 
 & 3 &  &  &  &                                                              & ✓ & <code>$238$</code> & <code>$238$</code><br><code>$1.00$</code> & <code>$0$</code><br><code>$0.00$</code> & ○ & --- & --- & --- \\ 
 & 4 &  &  &  &                                                              & ✓ & <code>$238$</code> & <code>$238$</code><br><code>$1.00$</code> & <code>$0$</code><br><code>$0.00$</code> & ○ & --- & --- & --- \\ 
 & 5 &  &  &  &                                                              & ✓ & <code>$238$</code> & <code>$238$</code><br><code>$1.00$</code> & <code>$0$</code><br><code>$0.00$</code> & ○ & --- & --- & --- \\ 
 & 6 &  &  &  &                                                              & ✓ & <code>$238$</code> & <code>$238$</code><br><code>$1.00$</code> & <code>$0$</code><br><code>$0.00$</code> & ○ & --- & --- & --- \\ 
 & 7 &  &  &  &                                                              & ✓ & <code>$238$</code> & <code>$238$</code><br><code>$1.00$</code> & <code>$0$</code><br><code>$0.00$</code> & ○ & --- & --- & --- \\ 
 & 8 &  &  &  &                                                              & ✓ & <code>$238$</code> & <code>$238$</code><br><code>$1.00$</code> & <code>$0$</code><br><code>$0.00$</code> & ○ & --- & --- & --- \\ 
 & 9 &  &  &  &                                                              & ✓ & <code>$238$</code> & <code>$238$</code><br><code>$1.00$</code> & <code>$0$</code><br><code>$0.00$</code> & ○ & --- & --- & --- \\ 
 & 10 &  &  &  &                                                              & ✓ & <code>$238$</code> & <code>$238$</code><br><code>$1.00$</code> & <code>$0$</code><br><code>$0.00$</code> & ○ & --- & --- & --- \\ 
 & 11 &  &  &  &                                                              & ✓ & <code>$238$</code> & <code>$238$</code><br><code>$1.00$</code> & <code>$0$</code><br><code>$0.00$</code> & ○ & --- & --- & --- \\ 
\bottomrule
\end{longtable}
\begin{minipage}{\linewidth}
2023-09-27 10:40:43 MDT
\textless{} 1 s
2023-09-27 10:40:44 MDT\\
\end{minipage}

\begin{Shaded}
\begin{Highlighting}[]
\NormalTok{all\_agents\_conditional[[}\DecValTok{3}\NormalTok{]]}
\end{Highlighting}
\end{Shaded}

\setlength{\LTpost}{0mm}
\begin{longtable}{lrlllccrrrcccc}
\caption*{
{\large Pointblank Validation} \\ 
{\small Conduct automated tests}
} \\ 
\toprule
 &  & STEP & COLUMNS & VALUES & TBL & EVAL & UNITS & PASS & FAIL & W & S & N & EXT \\ 
\midrule
 & 1 &  &  &  &                                                              & ✓ & <code>$1K$</code> & <code>$1K$</code><br><code>$1.00$</code> & <code>$0$</code><br><code>$0.00$</code> & ○ & --- & --- & --- \\ 
 & 2 &  &  &  &                                                              & ✓ & <code>$1K$</code> & <code>$1K$</code><br><code>$0.99$</code> & <code>$1$</code><br><code>$0.01$</code> & ● & --- & --- &  \\ 
 & 3 &  &  &  &                                                              & ✓ & <code>$1K$</code> & <code>$1K$</code><br><code>$0.99$</code> & <code>$2$</code><br><code>$0.01$</code> & ● & --- & --- &  \\ 
\bottomrule
\end{longtable}
\begin{minipage}{\linewidth}
2023-09-27 10:40:44 MDT
\textless{} 1 s
2023-09-27 10:40:44 MDT\\
\end{minipage}

We now have automated, different checks across multiple datasets!

We didn't check whether values fell within a certain range in the
\texttt{palmer\_penguins} data, we didn't check for duplicate IDs in our
\texttt{telemetry} data, and we performed both of those checks in the
\texttt{wellbeing} data.

We also caught duplicate IDs in both datasets where we checked, along
with a couple of columns with values way outside of our specified range
in \texttt{telemetry}.

\hypertarget{preprocessing-based-on-tests}{%
\section{Preprocessing Based on
Tests}\label{preprocessing-based-on-tests}}

Let's apply the necessary preprocessing to the datasets, also using
\texttt{pmap()} with preprocessing flags.

This applies the same logic as our initial checks but this time we'll do
a preprocessing step or not based on the flag.

For example, we won't run \texttt{distinct()} on the telemetry dataset
because we aren't concerned about duplicate IDs in that data.

\begin{Shaded}
\begin{Highlighting}[]
\NormalTok{all\_datasets\_processed }\OtherTok{\textless{}{-}} \FunctionTok{pmap}\NormalTok{(}\AttributeTok{.l =} \FunctionTok{list}\NormalTok{(}\AttributeTok{dataset\_for\_clean =}\NormalTok{ all\_datasets\_different, }\AttributeTok{id =}\NormalTok{ id\_var\_names, }
                                    \AttributeTok{clean\_dupe\_ids =}\NormalTok{ check\_dupe\_ids, }\AttributeTok{clean\_var\_range =}\NormalTok{ check\_var\_range,}
                                    \AttributeTok{vars\_value\_check =}\NormalTok{ vars\_value\_check),}
                               \AttributeTok{.f =} \ControlFlowTok{function}\NormalTok{(dataset\_for\_clean,}
\NormalTok{                                             id,}
\NormalTok{                                             clean\_dupe\_ids,}
\NormalTok{                                             clean\_var\_range,}
\NormalTok{                                             vars\_value\_check) \{}
    
\NormalTok{    dataset\_for\_clean }\SpecialCharTok{\%\textgreater{}\%}
\NormalTok{      \{}\ControlFlowTok{if}\NormalTok{ (clean\_dupe\_ids }\SpecialCharTok{==} \ConstantTok{TRUE}\NormalTok{)}
        \FunctionTok{distinct}\NormalTok{(.,}
          \FunctionTok{pick}\NormalTok{(}\FunctionTok{contains}\NormalTok{(id)), }\AttributeTok{.keep\_all =} \ConstantTok{TRUE}\NormalTok{)}
        \ControlFlowTok{else}\NormalTok{ .}
\NormalTok{        \} }\SpecialCharTok{\%\textgreater{}\%}
\NormalTok{      \{}\ControlFlowTok{if}\NormalTok{ (clean\_var\_range }\SpecialCharTok{==} \ConstantTok{TRUE}\NormalTok{)}
        \FunctionTok{mutate}\NormalTok{(.,}
               \FunctionTok{across}\NormalTok{(}
                 \AttributeTok{.cols =} \FunctionTok{c}\NormalTok{(}\SpecialCharTok{!!!}\NormalTok{vars\_value\_check),}
                 \AttributeTok{.fns =} \SpecialCharTok{\textasciitilde{}}\FunctionTok{case\_when}\NormalTok{(}
\NormalTok{                   .x }\SpecialCharTok{\textgreater{}} \DecValTok{4} \SpecialCharTok{\textasciitilde{}} \FloatTok{4.0}\NormalTok{,}
\NormalTok{                   .x }\SpecialCharTok{\textless{}} \SpecialCharTok{{-}}\DecValTok{4} \SpecialCharTok{\textasciitilde{}} \SpecialCharTok{{-}}\FloatTok{4.0}\NormalTok{,}
                   \ConstantTok{TRUE} \SpecialCharTok{\textasciitilde{}}\NormalTok{ .x}
\NormalTok{                 )}
\NormalTok{               )}
\NormalTok{        )}
        \ControlFlowTok{else}\NormalTok{ .}
\NormalTok{        \}}

  
\NormalTok{\})}
\end{Highlighting}
\end{Shaded}

And finally let's re-run our tests

\begin{Shaded}
\begin{Highlighting}[]
\NormalTok{all\_agents\_processed }\OtherTok{\textless{}{-}} \FunctionTok{pmap}\NormalTok{(}\AttributeTok{.l =} \FunctionTok{list}\NormalTok{(}\AttributeTok{dataset\_for\_check =}\NormalTok{ all\_datasets\_processed, }\AttributeTok{id =}\NormalTok{ id\_var\_names, }
                                    \AttributeTok{check\_dupe\_ids =}\NormalTok{ check\_dupe\_ids, }\AttributeTok{check\_var\_range =}\NormalTok{ check\_var\_range,}
                                    \AttributeTok{vars\_value\_check =}\NormalTok{ vars\_value\_check),}
                               \AttributeTok{.f =} \ControlFlowTok{function}\NormalTok{(dataset\_for\_check,}
\NormalTok{                                             id,}
\NormalTok{                                             check\_dupe\_ids,}
\NormalTok{                                             check\_var\_range,}
\NormalTok{                                             vars\_value\_check) \{}
    
\NormalTok{    dataset\_for\_check }\SpecialCharTok{\%\textgreater{}\%}
      \FunctionTok{create\_agent}\NormalTok{(}
        \AttributeTok{label =} \StringTok{"Conduct automated tests"}\NormalTok{,}
        \AttributeTok{actions =} \FunctionTok{action\_levels}\NormalTok{(}\AttributeTok{warn\_at =} \DecValTok{1}\NormalTok{)}
\NormalTok{      ) }\SpecialCharTok{\%\textgreater{}\%}
\NormalTok{      \{}\ControlFlowTok{if}\NormalTok{ (check\_dupe\_ids }\SpecialCharTok{==} \ConstantTok{TRUE}\NormalTok{)}
        \FunctionTok{rows\_distinct}\NormalTok{(.,}
          \AttributeTok{columns =} \FunctionTok{vars}\NormalTok{(}\SpecialCharTok{!!!}\NormalTok{id)}
\NormalTok{          )}
        \ControlFlowTok{else}\NormalTok{ .}
\NormalTok{        \} }\SpecialCharTok{\%\textgreater{}\%}
\NormalTok{      \{}\ControlFlowTok{if}\NormalTok{ (check\_var\_range }\SpecialCharTok{==} \ConstantTok{TRUE}\NormalTok{)}
\NormalTok{        pointblank}\SpecialCharTok{::}\FunctionTok{col\_vals\_between}\NormalTok{(.,}
          \AttributeTok{columns =}\NormalTok{ vars\_value\_check,}
          \AttributeTok{left =} \SpecialCharTok{{-}}\DecValTok{4}\NormalTok{,}
          \AttributeTok{right =} \DecValTok{4}\NormalTok{,}
          \AttributeTok{na\_pass =} \ConstantTok{TRUE}
\NormalTok{        )}
        \ControlFlowTok{else}\NormalTok{ .}
\NormalTok{        \} }\SpecialCharTok{\%\textgreater{}\%} 
      \FunctionTok{interrogate}\NormalTok{()}

  
\NormalTok{\})}
\end{Highlighting}
\end{Shaded}

\begin{Shaded}
\begin{Highlighting}[]
\NormalTok{all\_agents\_processed[[}\DecValTok{1}\NormalTok{]]}
\end{Highlighting}
\end{Shaded}

\setlength{\LTpost}{0mm}
\begin{longtable}{lrlllccrrrcccc}
\caption*{
{\large Pointblank Validation} \\ 
{\small Conduct automated tests}
} \\ 
\toprule
 &  & STEP & COLUMNS & VALUES & TBL & EVAL & UNITS & PASS & FAIL & W & S & N & EXT \\ 
\midrule
 & 1 &  &  & — &                                                              & ✓ & <code>$190$</code> & <code>$190$</code><br><code>$1.00$</code> & <code>$0$</code><br><code>$0.00$</code> & ○ & --- & --- & --- \\ 
\bottomrule
\end{longtable}
\begin{minipage}{\linewidth}
2023-09-27 10:40:47 MDT
\textless{} 1 s
2023-09-27 10:40:47 MDT\\
\end{minipage}

\begin{Shaded}
\begin{Highlighting}[]
\NormalTok{all\_agents\_processed[[}\DecValTok{2}\NormalTok{]]}
\end{Highlighting}
\end{Shaded}

\setlength{\LTpost}{0mm}
\begin{longtable}{lrlllccrrrcccc}
\caption*{
{\large Pointblank Validation} \\ 
{\small Conduct automated tests}
} \\ 
\toprule
 &  & STEP & COLUMNS & VALUES & TBL & EVAL & UNITS & PASS & FAIL & W & S & N & EXT \\ 
\midrule
 & 1 &  &  & — &                                                              & ✓ & <code>$236$</code> & <code>$236$</code><br><code>$1.00$</code> & <code>$0$</code><br><code>$0.00$</code> & ○ & --- & --- & --- \\ 
 & 2 &  &  &  &                                                              & ✓ & <code>$236$</code> & <code>$236$</code><br><code>$1.00$</code> & <code>$0$</code><br><code>$0.00$</code> & ○ & --- & --- & --- \\ 
 & 3 &  &  &  &                                                              & ✓ & <code>$236$</code> & <code>$236$</code><br><code>$1.00$</code> & <code>$0$</code><br><code>$0.00$</code> & ○ & --- & --- & --- \\ 
 & 4 &  &  &  &                                                              & ✓ & <code>$236$</code> & <code>$236$</code><br><code>$1.00$</code> & <code>$0$</code><br><code>$0.00$</code> & ○ & --- & --- & --- \\ 
 & 5 &  &  &  &                                                              & ✓ & <code>$236$</code> & <code>$236$</code><br><code>$1.00$</code> & <code>$0$</code><br><code>$0.00$</code> & ○ & --- & --- & --- \\ 
 & 6 &  &  &  &                                                              & ✓ & <code>$236$</code> & <code>$236$</code><br><code>$1.00$</code> & <code>$0$</code><br><code>$0.00$</code> & ○ & --- & --- & --- \\ 
 & 7 &  &  &  &                                                              & ✓ & <code>$236$</code> & <code>$236$</code><br><code>$1.00$</code> & <code>$0$</code><br><code>$0.00$</code> & ○ & --- & --- & --- \\ 
 & 8 &  &  &  &                                                              & ✓ & <code>$236$</code> & <code>$236$</code><br><code>$1.00$</code> & <code>$0$</code><br><code>$0.00$</code> & ○ & --- & --- & --- \\ 
 & 9 &  &  &  &                                                              & ✓ & <code>$236$</code> & <code>$236$</code><br><code>$1.00$</code> & <code>$0$</code><br><code>$0.00$</code> & ○ & --- & --- & --- \\ 
 & 10 &  &  &  &                                                              & ✓ & <code>$236$</code> & <code>$236$</code><br><code>$1.00$</code> & <code>$0$</code><br><code>$0.00$</code> & ○ & --- & --- & --- \\ 
 & 11 &  &  &  &                                                              & ✓ & <code>$236$</code> & <code>$236$</code><br><code>$1.00$</code> & <code>$0$</code><br><code>$0.00$</code> & ○ & --- & --- & --- \\ 
\bottomrule
\end{longtable}
\begin{minipage}{\linewidth}
2023-09-27 10:40:47 MDT
\textless{} 1 s
2023-09-27 10:40:47 MDT\\
\end{minipage}

\begin{Shaded}
\begin{Highlighting}[]
\NormalTok{all\_agents\_processed[[}\DecValTok{3}\NormalTok{]]}
\end{Highlighting}
\end{Shaded}

\setlength{\LTpost}{0mm}
\begin{longtable}{lrlllccrrrcccc}
\caption*{
{\large Pointblank Validation} \\ 
{\small Conduct automated tests}
} \\ 
\toprule
 &  & STEP & COLUMNS & VALUES & TBL & EVAL & UNITS & PASS & FAIL & W & S & N & EXT \\ 
\midrule
 & 1 &  &  &  &                                                              & ✓ & <code>$1K$</code> & <code>$1K$</code><br><code>$1.00$</code> & <code>$0$</code><br><code>$0.00$</code> & ○ & --- & --- & --- \\ 
 & 2 &  &  &  &                                                              & ✓ & <code>$1K$</code> & <code>$1K$</code><br><code>$1.00$</code> & <code>$0$</code><br><code>$0.00$</code> & ○ & --- & --- & --- \\ 
 & 3 &  &  &  &                                                              & ✓ & <code>$1K$</code> & <code>$1K$</code><br><code>$1.00$</code> & <code>$0$</code><br><code>$0.00$</code> & ○ & --- & --- & --- \\ 
\bottomrule
\end{longtable}
\begin{minipage}{\linewidth}
2023-09-27 10:40:47 MDT
\textless{} 1 s
2023-09-27 10:40:47 MDT\\
\end{minipage}

\hypertarget{finishing-the-pipeline-by-writing-the-new-datasets-to-processed-data}{%
\section{Finishing the Pipeline by Writing the New Datasets to Processed
Data\ldots{}}\label{finishing-the-pipeline-by-writing-the-new-datasets-to-processed-data}}

\begin{Shaded}
\begin{Highlighting}[]
\CommentTok{\# Write way less per path}

\NormalTok{file\_names }\OtherTok{\textless{}{-}} \FunctionTok{c}\NormalTok{(}\StringTok{"palmer\_penguins\_improved"}\NormalTok{,}
                \StringTok{"wellbeing"}\NormalTok{,}
                \StringTok{"telemetry"}\NormalTok{)}

\CommentTok{\# Start with an empty list of file paths}

\NormalTok{output\_paths\_processed }\OtherTok{\textless{}{-}} \FunctionTok{list}\NormalTok{()}

\CommentTok{\# Create the for loop that will create a populated list of file paths}

\ControlFlowTok{for}\NormalTok{ (file\_name }\ControlFlowTok{in}\NormalTok{ file\_names) \{}
  
  \CommentTok{\# Create the string needed for the file path using \textasciigrave{}glue\textasciigrave{}}
  
\NormalTok{  path\_init }\OtherTok{\textless{}{-}} \FunctionTok{glue}\NormalTok{(}\StringTok{"data\_from\_gsheets/processed/\{file\_name\}.csv"}\NormalTok{)}
  
  \CommentTok{\# Create the path itself using \textasciigrave{}here\textasciigrave{}}
  \CommentTok{\# Note, we could do this all at once but it\textquotesingle{}s less readable}
  
\NormalTok{  path }\OtherTok{\textless{}{-}}\NormalTok{ here}\SpecialCharTok{::}\FunctionTok{here}\NormalTok{(path\_init)}
  
  \CommentTok{\# Append (or add to the end) the current path to the formerly empty list}
  
\NormalTok{  output\_paths\_processed }\OtherTok{\textless{}{-}} \FunctionTok{append}\NormalTok{(output\_paths\_processed, path)}
  
\NormalTok{\}}
\end{Highlighting}
\end{Shaded}

\begin{Shaded}
\begin{Highlighting}[]
\FunctionTok{map2}\NormalTok{(all\_datasets\_processed, output\_paths\_processed, }\SpecialCharTok{\textasciitilde{}}\FunctionTok{write\_csv}\NormalTok{(.x, .y))}
\end{Highlighting}
\end{Shaded}

\hypertarget{and-writing-automated-reports-to-their-folder}{%
\section{And Writing Automated Reports to Their
Folder}\label{and-writing-automated-reports-to-their-folder}}

\begin{Shaded}
\begin{Highlighting}[]
\CommentTok{\# Start with an empty list of file paths}

\NormalTok{report\_paths\_processed }\OtherTok{\textless{}{-}} \FunctionTok{list}\NormalTok{()}

\CommentTok{\# Create the for loop that will create a populated list of file paths}

\ControlFlowTok{for}\NormalTok{ (file\_name }\ControlFlowTok{in}\NormalTok{ file\_names) \{}
  
  \CommentTok{\# Create the string needed for the file path using \textasciigrave{}glue\textasciigrave{}}
  
\NormalTok{  path\_init }\OtherTok{\textless{}{-}} \FunctionTok{glue}\NormalTok{(}\StringTok{"reports/processed/\{now()\}\_\{file\_name\}\_report.html"}\NormalTok{)}
  
  \CommentTok{\# Create the path itself using \textasciigrave{}here\textasciigrave{}}
  \CommentTok{\# Note, we could do this all at once but it\textquotesingle{}s less readable}
  
\NormalTok{  path }\OtherTok{\textless{}{-}}\NormalTok{ here}\SpecialCharTok{::}\FunctionTok{here}\NormalTok{(path\_init)}
  
  \CommentTok{\# Append (or add to the end) the current path to the formerly empty list}
  
\NormalTok{  report\_paths\_processed }\OtherTok{\textless{}{-}} \FunctionTok{append}\NormalTok{(report\_paths\_processed, path)}
  
\NormalTok{\}}
\end{Highlighting}
\end{Shaded}

\begin{Shaded}
\begin{Highlighting}[]
\CommentTok{\# Commented out here so we don\textquotesingle{}t create a new version of the report during every reload}

\CommentTok{\# map2(all\_agents\_processed, report\_paths\_processed, \textasciitilde{}export\_report(.x, .y))}
\end{Highlighting}
\end{Shaded}

\hypertarget{conclusion}{%
\section{Conclusion}\label{conclusion}}

We now have all the building blocks necessary to create a scalable,
reproducible pipeline with automated checks across all datasets from a
single type of source. \texttt{\{pointblank\}} has a dizzying number of
kinds of tests you can perform on your data, and you can learn more from
\href{https://rstudio.github.io/pointblank/index.html}{its docs}.

Many organizations will understandably stop here. They get there data
from one kind of source\footnote{like Qualtrics} and will plan to run
this pipeline manually whenever a lot of new data comes in. They'll also
add new checks as they go.

\begin{itemize}
\tightlist
\item
  However, there are still three more advanced topics this resource
  might cover:

  \begin{itemize}
  \tightlist
  \item
    Automated preprocessing based on failed checks + retesting after
    that preprocessing
  \item
    Getting data from multiple kinds of sources before testing +
    preprocessing
  \item
    Automatically running your data pipeline on a schedule.
  \end{itemize}
\item
  If there's massive demand and I have more time than I anticipate, I
  might:

  \begin{itemize}
  \tightlist
  \item
    Add a case study for data engineering in R
  \item
    Create a companion version of this textbook using Python + SQL to
    help people bridge their data engineering skills to the far more
    commonly used tools at larger scales.
  \end{itemize}
\end{itemize}



\end{document}
